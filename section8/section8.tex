\section{Ecosistemi Basati su IoTeX}

La blockchain IoTeX supporta una varietà di ecosistemi IoT: shared economy, smart home, veicoli autonomi, supply chain, ecc. Diversi tipi di sviluppatori possono sfruttare IoTeX in modi diversi. Gli sviluppatori supportati da IoTeX includono produttori di hardware IoT, sviluppatori di sistemi di controllo dei dispositivi IoT, sviluppatori di app per smart home, produttori di dispositivi per la shared economy, integratori di dati della supply chain, venditori di data-crowdsourcing, sviluppatori di automobili a guida autonoma, ecc. Questa sezione descrive alcuni ecosistemi basati su IoTeX.

\subsection{Shared Economy}
Negli ultimi anni, molte aziende si sono concentrate sulla shared economy, dalla condivisione dei viaggi realizzata da Uber/Lyft/Didi, alla condivisione di alloggi di Airbnb, alla condivisione di biciclette di Mobike/ofo, alla condivisione di piccoli oggetti come power bank, ombrelli, ecc... Tutti migliorano la vita delle persone, anche se alcuni business ne ricevono un danno. Discutere questi modelli di business non è oggetto di questo documento: qui ci concentriamo principalmente sulla loro architettura tecnologica. Tra tutte le economie condivise, la condivisione degli spostamenti in auto è l'unico che al momento non può fare a meno del lavoro dell'uomo, ovvero del conducente: non è un'economia basata su IoT. Tuttavia, in futuro, quando la tecnologia delle auto a guida autonoma sarà matura e diffusa, la condivisione dei viaggi sarà anch'essa basato su IoT.

Le economie condivise basate su IoT hanno alcune somiglianze tra loro: tutte richiedono un canone di locazione ed una qualche forma di "serratura" che può essere "sbloccata" da una cauzione. È assolutamente possibile ed anche efficiente basare l'intero processo di condivisione e Attualmente, queste economie sono basate su un cloud centralizzato, e questo porta a vari inconvenienti:

\begin{enumerate}
    \item Un ingente deposito cauzionale è detenuto da una società che potrebbe non essere affidabile. Di recente ci sono stati molti casi in cui l'azienda che gestiva un servizio di bici condivise in Cina non è stata in grado di restituire i depositi ai propri clienti;

    \item Le economie condivise non sono interamente gestite dalla comunità. Molti oggetti condivisi sono di proprietà di un'azienda e ciò ha causato uno spreco di risorse. Prendiamo le biciclette condivise come esempio: quando le aziende di condivisione bici chiudono l'attività, quelle bici vengono eliminate.

    \item A causa della loro natura centralizzata, i dati dell'utente saranno archiviati e controllati da una società. Ci sono rischi che il server o il client delle società possano essere violati al fine di ottenere i dati degli utenti.
\end{enumerate}

IoTeX, come infrastruttura, potrebbe essere utilizzato per creare queste applicazioni senza i problemi di cui sopra, e rendere le economie condivise decentralizzate e più efficienti. In concreto, un'economia condivisa basata su IoTeX offre i seguenti vantaggi:

\begin{figure}[ht]
    \centering
    \includegraphics[width=\textwidth]{Figura7}
    \caption{Shared Economy basata su IoTeX}
    \label{fig:Figura7}
\end{figure}

\begin{enumerate}
    \item Il deposito cauzionale è interamente regolato da uno smart contract. Poiché nessuno trattiene i soldi, la restituzione del deposito è sempre garantito. Gli utenti non sono obbligati a fidarsi di una compagnia per utilizzare il servizio.

    \item Ogni oggetto condiviso realizza il suo valore e la sua missione autonomamente. Nell'ecosistema, non importa chi possiede gli oggetti in esso condivisi. Chiunque può possederli e contribuire all'ecosistema. L'economia condivisa può essere gestita dalla comunità. Di conseguenza, le aziende possono svolgere il ruolo di realizzare/manutenere i dispositivi IoT di blocco (la "serratura") e gestire la community. È un modello di business molto più leggero che le aziende possono espandere velocemente per servire più persone.

    \item Ancora una volta, gli utenti non devono fidarsi dell'azienda per mantenere i propri dati. I dati sono mantenuti nella blockchain IoTeX, con anche la protezione della privacy degli utenti.
\end{enumerate}

La Figura \ref{fig:Figura7} descrive come funziona la shared economy basata sulla blockchain IoTeX.

\subsection{Smart Home}

Nel mercato corrente della smart home, molti produttori di dispositivi IoT continuano a utilizzare tecnologie obsolete per sviluppare i loro prodotti. Hanno bisogno di una grande quantità di lavoro di sviluppo per i loro cloud. Il costo di sviluppo e manutenzione è elevato, e le prestazioni sono basse a causa del tragitto di andata/ritorno richiesto verso/dal cloud. Distribuendo i loro prodotti sulla blockchain IoTeX, il produttore ridurrà in gran parte i costi operativi di ingegnerizzazione e di cloud computing e, allo stesso tempo, aumenterà notevolmente le prestazioni dei loro dispositivi. Nel semplice esempio di una lampadina intelligente, con la tecnologia cloud, occorrono due tragitti dei dati dall'istante in cui l'utente comanda di cambiare lo stato della lampadina. I produttori di hardware non sono esperti di cloud, così spesso il loro servizio non è ottimale: la comunicazione, tra l'andata e il ritorno, può durare da uno a tre secondi. Ciò li costringe a utilizzare i servizi cloud di grandi aziende IT, e ci sono due aspetti negativi dell'utilizzo di questi servizi cloud:

\begin{enumerate}
    \item I produttori di hardware IoT non possono controllare completamente la disponibilità dei servizi cloud.

    \item Devono di pagare continuamente per il servizio cloud, a fronte di un incasso una-tantum per la vendita dei loro dispositivi IoT.

    \item Ci sono rischi di hacking del loro cloud: in caso di hacking lato client o intranet i dati degli utenti verrebbero trafugati o addirittura si potrebbero creare problemi di sicurezza domestica.
\end{enumerate}

Al contrario, la blockchain IoTeX gestisce i dispositivi localmente, e interagisce con la blockchain pubblica su internet solo quando necessario. La blockchain pubblica è gestita dalla comunità. Non ci sono costi di manutenzione per i produttori di IoT. La blockchain di IoTeX dispone di protezione della privacy, il che può impedire la rivelazione di dati sensibili oppure che l'unità di controllo venga hackerata, anche nel caso che la rete intranet dell'utente non sia sicura.

\begin{figure}
    \centering
    \includegraphics[width=\textwidth]{Figura8}
    \caption{Smart Home basata su IoTeX}
    \label{fig:Figura8}
\end{figure}

Oltre a consentire ai produttori di IoT di implementare i loro dispositivi sulla propria blockchain, IoTeX collaborerà con i produttori di chip IoT per sviluppare chip abilitati all'uso della blockchain IoTeX, per accelerare il ciclo di progettazione e produzione di dispositivi IoT. I produttori di dispositivi IoT potranno semplicemente integrare il chip per fare in modo che i loro dispositivi siano immediatamente supportati dalla blockchain IoTeX.


\subsection{Gestione delle identità}
Il mondo in crescita dell'IoT ha avuto un impatto su come la gestione delle identità e degli accessi (\emph{Identity and Access Management} - IAM) funzionerà. In termini di identità delle cose, lo IAM deve essere in grado di gestire il sistema utente-dispositivo, dispositivo-dispositivo, e/o dispositivo-servizio. Un modo semplice per la gestione dell'identità è quello di considerare la blockchain IoTeX come un sistema PKI decentralizzato (grazie alla sua immutabilità), in cui a ciascuna entità viene rilasciata un'identità crittografica sotto forma di certifcato TLS con la chiave privata corrispondente. Questo certificato, che tendenzialmente è di breve durata, viene firmato dal certificato di lunga durata integrato nel dispositivo, e poi pubblicato sulla blockchain IoTeX (rootchain o subchain). I nodi ed altre entità possono accedere e fidarsi del certificato di breve durata ancorato alla
blockchain, e i dispositivi possono quindi autenticarsi quando vanno online, garantendo sicurezza di comunicazione con gli altri dispositivi, servizi e utenti, e dimostrare la loro integrità.

Inoltre, è possibile organizzare gerarchicamente i certificati di lunga durata integrati nei dispositivi, come per la PKI convenzionale, in cui i dispositivi genitore possono firmare i certificati dei dispositivi figli. Grazie alla gerarchia, diventa possibile la revoca e la rotazione dei certificati. Ad esempio, se un dispositivo viene compromesso, il suo dispositivo genitore o anche il dispositivo nonno potrebbero firmare un comando di revoca e inviarlo alla blockchain dove quest'ultimo invalida il certificato del dispositivo compromesso.