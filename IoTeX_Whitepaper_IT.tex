\documentclass[a4paper,12pt]{article}
\usepackage{array}
\usepackage{enumitem}
\usepackage[a4paper]{geometry}
\usepackage[english,italian]{babel}
\usepackage{amsmath}
\usepackage{dsfont}
\usepackage{graphicx}
\usepackage{hyperref}
\usepackage{url}
\usepackage[utf8]{inputenc}
\usepackage{bookmark}

\geometry{left=33mm, right=30mm}

\title{
	IoTeX
	\linebreak
	\Large Una Rete Decentralizzata per l'Internet of Things
	\linebreak
	\Large Basata su una Blockchain Incentrata sulla Privacy}
	\author{Il Team IoTeX (support@iotex.io)}
	\date{Ultimo Aggiornamento: 12 Luglio, 2018
	\linebreak Version 1.5
	}


\begin{document}

\maketitle

\vspace{200pt}


\textbf{Dichiarazione di limitazione di responsabilità.} Questo documento deve essere inteso come una panoramica tecnica. Non vuole essere esaustivo, né rappresentare un progetto definitivo; pertanto aspetti secondari, come API, interconnessioni o linguaggi di programmazione, non vengono trattati.

\pagebreak

\begin{abstract}
	La maggior parte dei dispositivi IoT (Internet of Things), sebbene decentralizzati per natura, ad oggi vengono distribuiti in modo centralizzato. Molti problemi sono emersi: scalabilità, costi operativi elevati, problemi di privacy, rischi per la sicurezza, e mancanza di valore funzionale. La Blockchain, decentralizzata per definizione, può rappresentare una buona soluzione a questi problemi. Innanzitutto, la blockchain è abbastanza elastica da risolvere la sfida della scalabilità dell'IoT in modo economicamente vantaggioso. In secondo luogo, mantenendo i dati all'interno di blockchain ben definite, si eliminano i timori per i dati IoT memorizzati in cloud, potenzialmente suscettibili di trapelare o di essere violati. In terzo luogo, le blockchain con smart contract e token hanno un enorme potenziale per consentire il coordinamento autonomo dei dispositivi al fine di creare valore funzionale. Tuttavia, le blockchain esistenti hanno i loro limiti nell'affrontare i problemi dell'IoT, a causa delle caratteristiche peculiari che lo contraddistinguono, ad esempio la grande quantità e l'eterogeneità dei dispositivi, i limiti nella potenza di elaborazione, nell'archiviazione dati, nell'alimentazione, ecc.
	Questo documento introduce IoTeX, una rete decentralizzata per l'IoT basata su una blockchain incentrata sulla privacy, con quattro importanti innovazioni:

	\begin{itemize}

		\item
		      Blockchain in blockchain, per una rete distribuita ben bilanciata che massimizza la scalabilità e la privacy in modo economicamente vantaggioso;

		\item
		      Privacy reale integrata sulla blockchain, basata sui meccanismi di \emph{codice di pagamento inoltrabile}, \emph{ring signature a dimensione costante} e senza configurazione \emph{trusted}, e una implementazione iniziale del \emph{Bulletproof};

		\item
		      Consenso rapido con finalità immediata, per migliorare notevolmente l'efficienza della rete, e ridurre i costi di transazione;

		\item
		      Architetture flessibili e leggere, per la realizzazione delle applicazioni IoT chiave in molteplici settori industriali.

	\end{itemize}

\end{abstract}

\pagebreak

\tableofcontents

\pagebreak

\section{L'Internet of Things}
L'Internet of Things (IoT) sta emergendo rapidamente come manifestazione della visione di una società collegata in rete: qualsiasi cosa che può beneficiare di una connessione, viene connesso. Eppure, questa trasformazione su vasta scala rappresenta solo l'inizio. Il numero di dispositivi IoT è destinato a crescere del 21\% ogni anno, raggiungendo i 18 miliardi nel 2022 \cite{c10}, mentre il mercato globale dell'IoT è destinato a passare dai 170 miliardi di dollari del 2017 a 560 miliardi di dollari entro il 2022 \cite{c15}, con un tasso di crescita annua del 26,9\%. Sebbene molti esperti dell'industria e consumatori entusiasti hanno definito l'IoT come la prossima rivoluzione industriale o il prossimo internet, ci sono tre problemi principali che frenano in maniera massiccia lo sviluppo e l'adozione dell'IoT.

\subsection{Il problema della scalabilità}
La maggior parte dei dispositivi IoT sono ad oggi connessi e controllati in maniera centralizzata. I dispositivi IoT sono connessi ad infrastrutture di back-end, su servizi cloud pubblici oppure localmente all'interno di server farm, per trasmettere dati oppure ricevere comandi di controllo.
Attualmente, la dimensione dell IoT è strozzata dal livello di scalabilità ed elasticità di queste infrastrutture di back-end, server e data center. E' improbabile che il costo operativo sostanzialmente elevato necessario per scalare l'IoT sia coperto dai profitti della vendita dei dispositivi. Di conseguenza, molti fornitori IoT non riescono a proporre dispositivi economicamente vantaggiosi ed applicazioni che siano abbastanza scalabili ed affidabili per scenari reali.

\subsection{Mancanza di Privacy}
Si prevede che l'IoT permetterà la partecipazione di massa degli utenti finali a servizi mission critical come l'energia, la  mobilità, la stabilità legale e democratica. Le sfide per la privacy hanno origine dal fatto che l'IoT interagisce con il mondo fisico in modi diretti e automatici, e la quantità di dati raccolti aumenterà notevolmente man mano che si diffonderà. Alcune  delle minacce alla privacy comuni, come elencate in \cite{c37}, sono:

\begin{enumerate}

	\item
	      Identificazione: associare un identificatore  persistente, ad es. un nome e un indirizzo o uno pseudonimo di qualsiasi tipo, con un individuo;
	\item
	      Localizzazione e tracciamento: ottenere la posizione di un individuo attraverso diversi mezzi;
	\item
	      Profilazione: Compilare fascicoli informativi sulle persone per dedurrne gli interessi per
	      associazione con altri profili e fonti di dati;
	\item
	      Interazione e presentazione che violano la privacy: trasmettere informazioni private attraverso un mezzo pubblico e nel processo rivelarle ad un pubblico indesiderato;
	\item
	      Transizioni del ciclo di vita: i dispositivi spesso memorizzano enormi quantità di dati sulla propria storia durante l'intero ciclo di vita, che potrebbero trapelare durante i cambiamenti della sfera di controllo nel ciclo di vita di un dispositivo;
	\item
	      Attacco all'inventario: raccolta non autorizzata di informazioni sull'esistenza e sulle caratteristiche degli oggetti personali, ad esempio, i ladri d'appartamento potrebbero utilizzare l'inventario dati per controllare la proprietà, e individuare un momento sicuro per entrare;
	\item
	      Collegamento: collegamento di sistemi diversi precedentemente separati in modo tale che la combinazione delle fonti di dati riveli informazioni (vere o errate) che il soggetto non aveva rivelato alle fonti isolate e, soprattutto, che non intendeva rivelare.
\end{enumerate}
Tutte queste tipiche minacce alla privacy sono dovute a perdite di dati a livello del dispositivo, oppure alla perdita di dati durante la comunicazione, o più spesso alla perdita di dati nella parte centralizzata della rete.

\subsection{Mancanza di Valore Funzionale}
La maggior parte delle soluzioni IoT esistenti non crea valore significativo. Il semplice fatto di "essere connesso" rappresenta al momento la proposta di valore più utilizzata. Tuttavia, abilitarne semplicemente la connettività non rende un dispositivo intelligente, o utile. La maggior parte del valore che l'IoT produce è dovuto all'interazione, alla cooperazione, ed infine al coordinamento autonomo di entità eterogenee. Alcune buone analogie sono: le singole cellule che cooperano per costruire gli organismi multicellulari, gli insetti che insieme costruiscono società, oppure gli uomini che costituiscono città e stati. Grazie alla cooperazione, tutti questi individui si uniscono per costruire qualcosa che ha un valore maggiore rispetto a tutti loro presi isolatamente. Sfortunatamente, secondo \cite{c29}, l'85\% dei dispositivi obsoleti non ha la capacità di interagire o cooperare con altri, a causa di problemi di compatibilità. La condivisione dei dati e le indicazioni operative per il business sono quasi impossibili.


\section{Blockchain}
La tecnologia della Blockchain è stata introdotta nel 2008 e la sua prima implementazione, ovvero Bitcoin, è stata introdotta un anno dopo, nel 2009, pubblicata nel documento \emph{Bitcoin: A Peer-to-Peer Electronic Cash System} \cite{c21} di Satoshi Nakamoto (pseudonimo). Essenzialmente, la blockchain è un database transazionale distribuito, condiviso tra tutti i nodi partecipanti nella rete. Questa è la principale innovazione tecnica di Bitcoin, ed agisce come un registro pubblico per le transazioni. Ogni nodo nel sistema ha una copia completa dello stato attuale della blockchain, che contiene ogni transazione che sia mai stata eseguita. Ogni blocco della blockchain contiene l'hash del blocco precedente, il che collega i due blocchi insieme. Tutti i nodi collegati tra loro diventano una blockchain.

\subsection{Ingredienti}
Una blockchain può essere percepita come un continuum tetra-dimensionale avente tre livelli orizzontali composti da transazioni e blocchi, consenso, e interfaccia di calcolo; e l'unico livello verticale di amministrazione (\emph{governance}).

\subsubsection{Transazioni e blocchi}
Trovandosi al livello orizzontale più in basso, le transazioni firmate vengono trasmesse tra tutti i nodi, mentre i blocchi vengono generati solo dai nodi completi (\emph{full nodes}). Questa è la base della blockchain, dove il trasferimento di beni digitali (e dunque il valore a loro associato) e la sicurezza degli account si ottengono attraverso primitive come la firma a curva ellittica, la funzione hash e il Merkle tree.

\subsubsection{Consenso}
Il livello orizzontale intermedio mostra la natura peer-to-peer della blockchain, dove tutti i nodi all'interno della rete raggiungono il consenso su tutti gli stati interni della catena attraverso tecniche come Proof of Work (PoW), Proof of Stake (PoS) e le loro varianti; Byzantine fault tolerance (BFT) e le sue varianti, \emph{etc}. Il livello di consenso interessa principalmente la scalabilità. Il PoW viene tipicamente considerato meno scalabile rispetto al PoS. Inoltre, questo livello ha un forte impatto sulla sicurezza in termini di doppia spesa ed altri attacchi che si concentrano sul modificare gli stati della blockchain in modi inattesi.

\subsubsection{Interfaccia di calcolo}
I primi due livelli orizzontali realizzano \emph{la forma} di una blockchain, mentre il livello di interfaccia di calcolo è fondamentale per garantire l'\emph{utilità} di una blockchain, il che comprende l'estendibilità ed l'usabilità. Ad esempio, Ethereum ha implementato gli smart contract per consentire la programmabilità, in modo da poter disporre di un \emph{"computer globale"} distribuito per l'esecuzione dei termini di un contratto. Anche il sidechain, insieme con il mining congiunto, sono stati sviluppati in modo intensivo per supportare la programmabilità. All'interno di protocolli di secondo livello come la rete Raiden \cite{c25}, è stato sviluppato il canale di stato per estendere la scalabilità della blockchain. Inoltre, anche gli strumenti, gli SDK, i framework e le interfacce grafiche sono estremamente importanti per l'usabilità. Il livello di interfaccia di calcolo offre agli sviluppatori la possibilità di sviluppare app decentralizzate (DApps), una funzionalità essenziale per rendere la blockchain utile e di valore.

\subsubsection{Amministrazione}
Come accade per gli organismi viventi, le blockchain di maggior successo saranno quelle che in futuro riusciranno ad adattarsi meglio al loro ambiente. Supponendo che questi sistemi debbano evolversi per sopravvivere, il progetto iniziale è importante ma, nel lungo termine, i meccanismi per il cambiamento lo saranno di più: essi sono noti come il livello verticale di amministrazione (\emph{governance}). Ci sono due componenti critici della governance:
\begin{itemize}
	\item
	      Incentivo: ogni gruppo nel sistema ha i propri incentivi. Gli incentivi non sono sempre allineati al 100\% con quelli di tutti gli altri gruppi nel sistema. I gruppi proporranno nel tempo cambiamenti che sono vantaggiosi per loro. Gli organismi sono "di parte" quando si tratta della propria sopravvivenza. Ciò normalmente si manifesta in cambiamenti nella struttura retributiva, nella politica monetaria o negli equilibri di potere.

	\item
	      Coordinamento: poiché è improbabile che tutti i gruppi risultino completamente allineati sugli incentivi in ogni momento, la capacità di ciascun gruppo di coordinarsi attorno agli incentivi comuni è fondamentale per loro per produrre un cambiamento. Se un gruppo riesce a coordinarsi meglio di un altro, crea uno squilibrio di potere a proprio favore. In pratica, un fattore decisivo per la sopravvivenza di una blockchain è quanto coordinamento si può realizzare utilizzando la blockchain (ad es. votando le regole del sistema come in Tezos \cite{c34}, o addirittura ripristinando uno stato precedente della blockchain se gli azionisti di maggioranza non gradiscono una modifica), rispetto al coordinamento che deve essere realizzato fuori dalla blockchain (come ad es. i Bitcoin Improvement Proposals (BIPs)
	      \cite{c3}).

\end{itemize}

\subsection{Modelli Operazionali}
Le blockchain possono essere categorizzate come "non autorizzate" ed "autorizzate", a seconda di come sono gestite. Ad esempio, Bitcoin è priva di autorizzazione, il che significa che chiunque può creare un indirizzo e iniziare a interagire con la rete: in questo caso si parla di "costruire fiducia in mancanza di affidabilità". Al contrario, la blockchain autorizzata è un ecosistema chiuso e monitorato, dove l'accesso di ciascun partecipante è defnito, e differenziato in base al suo ruolo: in questi casi si parla di "costruire fiducia in bassa affidabilità".
Vi sono vantaggi e svantaggi in ciascun approccio. Ad ogni modo, tutte queste considerazioni si riducono a compromessi di progetto fondamentali tra fiducia, scalabilità, elaborazione e complessità. Ad esempio, Bitcoin ed Ethereum sono blockchain costruite su nodi non affidabili, perché la scalabilità è fortemente desiderata. Quindi, o è richiesta molta elaborazione (nel caso del PoW), oppure è necessario un meccanismo di consenso più sofisticato. Al contrario, Fabric \cite{c14} è una blockchain autorizzata in cui tutti i nodi sono considerati affidabili e hanno identità crittografiche, ad esempio, rilasciate grazie ai servizi di membri come il Public Key Infrastructure (PKI), il che li rende altamente scalabili con poca elaborazione e un meccanismo di consenso relativamente semplice.

\begin{table}[htp]%
	\caption{Proprietà delle Blockchain: Benefici per l'IoT}
	\label{table:BlockchainBenefits}\centering %
	\begin{tabular}{l|l}
		\hline
		Proprietà della Blockchain  & Benefici per l'IoT       \\
		\hline
		Decentralizzazione          & Stabilità, Privacy       \\
		Bizantine Fault Tolerance   & Disponibilità, Sicurezza \\
		Trasaprenza \& Immutabilità & Assicurazione di Fiduca  \\
		Programmabilità             & Estendibilità            \\
		\hline
	\end{tabular}
\end{table}


\section{Benefici e Sfide di Blockchain e IoT}
Sensazione e percezione, trasformazione, trasmissione ed elaborazione sono l'essenza delle entità più intelligenti su questo pianeta. Per l'IoT, mentre i livelli di sensazione e percezione è distribuito per definizione, gli ultimi due al momento non lo sono, e ciò è la fonte della maggior parte dei problemi di scalabilità, privacy ed estendibilità. Possiamo immaginare la tecnologia blockchain, se essa funge da spina dorsale e sistema nervoso dell'IoT, come il miglior candidato per affrontare i problemi specifici di questo mondo precedentemente menzionati.


\subsection{Benefici}
Abbracciando la tecnologia blockchain, l'IoT beneficia immediatamente dei seguenti aspetti, grazie alle proprietà della blockchain, che includono la decentralizzaizione, il Byzantine Fault Tolerance, la trasparenza e l'immutabilità. La Tabella 1 riassume come queste proprietà della blockchain beneficiano l'IoT.


\subsubsection{Decentralizzazione}
La decentralizzazione libera gli utenti e i dispositivi dal monitoraggio esteso controllato in modo centralizzato, dunque affrontando in parte i timori riguardanti la vita privata imposte dalle entità centralizzate che monopolizzano il mercato e cercano di capire ogni aspetto degli utenti o dei dispositivi a loro beneficio, ad es. per comunicazioni pubblicitarie. La decentralizzazione, nel contesto della criptoeconomia, indica anche \"elasticità\", spesso definita come \"il livello al quale un sistema è in grado di adattarsi alle variazioni del carico di lavoro mediante allocazione e deallocazione di risorse in maniera autonoma, in modo tale che in ogni momento nel tempo le risorse disponibili soddisfino il più possibile la domanda corrente\". Una blockchain con la sottostante criptoeconomia può essere progettata in modo sufficientemente elastico ed economicamente conveniente per gli scenari e le applicazioni IoT. Ad esempio, con gli  incentivi sufficienti per farlo, potrebbero attivarsi ulteriori nodi nella blockchain qualora la rete avesse abbastanza attività da elaborare.

\subsubsection{Byzantine Fault Tolerance (BFT)}
L'obiettivo delByzantine Fault Tolerance (BFT) è quello di difendersi dai guasti nei componenti
di un sistema che possono fallire in modo arbitrario, cioè non solo fermandosi o andando in crash, ma mediante l'elaborazione errata delle richieste, corrompendo il loro stato locale e/o producendo risultati errati o incoerenti. Il Byzantine Fault Tolerance modella gli ambienti del mondo reale in cui computer e reti possono comportarsi in modi imprevisti a causa di guasti dell'hardware, congestione della rete e disconnessioni, nonché attacchi malevoli. La proprietà del BFT può essere sfruttata per ottenere molte caratteristiche desiderate riguardanti la sicurezza nel contesto dell'IoT, ad esempio elimina gli attacchi del tipo man-in-the-middle (MITM) in quanto non esiste un singolo flusso di comunicazione che può essere intercettato e manomesso, e rende gli attacchi del tipo Denial of Service (Dos) quasi impossibili.

\subsubsection{Trasparenza e immutabilità}
La Blockchain fornisce la sicurezza crittografica che i dati all'interno della catena di blocchi sono sempre trasparenti e immutabili, il che può essere utile in molti scenari, ad esempio, ancorando stati del mondo IoT sulla blockchain al fine dell'auditing, dell'autenticazione notarile e l'analisi forense, gestione delle identità, autenticazione ed autorizzazione.

\subsubsection{Programmabilità}
Il Bitcoin è stato realizzato con un livello di programmabilità di base, per consentire ad una transazione di essere eseguita solo se il piccolo script sottostante viene eseguito correttamente. Ethereum migliora questa caratteristica
fornendo smart contract Turing-completi che vengono scritto in un linguaggio di programmazione di alto livello ed eseguiti in una piccola macchina virtuale nota come EVM. Questa programmabilità potrebbe e dovrebbe essere estesa ai dispositivi IoT, alcuni dei quali al momento dispongono solo di una logica semplice e già codificata, che non può essere ulteriormente programmata una sola volta consegnati.

\subsection{Sfide}
Beneficiare delle tipiche proprietà fornite dalle blockchain non significa che ogni blockchain è adatta per l'uso nell'IoT. In realtà, sembra che nessuna delle blockchain pubbliche esistenti possa essere applicata all'IoT poiché ci sono alcuni problemi difficili da affrontare.

\subsubsection{Garantire la privacy nativamente non basta}
Le garanzie sulla privacy, intrinseche della blockchain, possono solo aiutare ad affrontare il problema della privacy
nell'IoT nella misura in cui essa conserva i dati su un registro decentralizzato piuttosto che su server centralizzati, usando la pseudonimia. Tuttavia, se lo pseudonimo di un dispositivo venisse messo in relazione con la sua identità, tutto ciò che è mai stato fatto sotto quello pseudonimo sarà ora collegato a quel dispositivo.

\subsubsection{La "pallottola d'argento" tra le blockchain non esiste}
Come accennato in precedenza, L'IoT è un universo di sistemi e dispositivi eterogenei con differenti scopi e capacità. È impossibile trovare una "pallottola d'argento" tra le possibili di blockchain, una soluzione che si adatti alla maggior parte degli scenari. Ad esempio, una blockchain per il coordinamento di milioni di nodi IoT industriali dovrebbe concentrarsi sull'elevata scalabilità e sul volume delle transazioni, mentre una blockchain per il coordinamento di dispositivi domestici intelligenti dovrebbe concentrarsi sulla privacy e sull'estendibilità. A livello macroscopico, i dispositivi IoT, come una specie a sé stante, sono in continua evoluzione ad un ritmo molto veloce: nuove tecnologie vengono integrate, nuovi standard sviluppati nuovi dispositivi realizzati e con nuove funzionalità. D'altra parte, ad un livello microscopico, anche la capacità, lo scopo e l'ambiente operativo del singolo dispositivo IoT cambiano nel tempo.

\subsubsection{Le operazioni sulla blockchain sono pesanti}
Nel mondo IoT, molti dispositivi sono considerati nodi deboli perché essi sono:
\begin{itemize}
	\item
	      Incapaci di eseguire il "mining" basato su PoW a causa della limitata potenza di calcolo;
	\item
	      Incapaci di memorizzare grandi quantità di dati (ad es. a livello di gigabyte, se non di terabyte o di petabyte) a causa dei vincoli di archiviazione ed alimentazione;
	\item
	      Incapaci di verificare tutte le transazioni elaborando l'intera blockchain;
	\item
	      Incapaci di connettersi sempre con gli altri nodi, a seconda della disponibilità online e della qualità della connessione;
\end{itemize}
Pertanto, la maggior parte delle blockchain esistenti sono troppo pesanti per l'IoT.

\subsection{Lavori correlati}
IOTA, che è stata rilasciata di recente, è costruita sulla base di una tecnologia non convenzionale conosciuta come Tangle \cite{c24}. IOTA cerca di disaccoppiare il meccanismo di transizione dello stato da quello di normalizzazione del consenso, eliminando concetti come blocchi e catena. Al contrario, ch emitte le transazioni è anche lo stesso che le approva e la verifica delle transazione viene realizzata utilizzando un grafico aciclico diretto (DAG) per effettuare la transazione in modo veloce e a costo zero. L'efficienza si ottiene grazie alla perdita stati definiti globalmente, il che rende funzionalità desiderabili quali il Simple Payment Verifcation (SPV) per i client leggeri, e gli smart contract abbastanza impegnativi. IoT Chain (ITC) \cite{c16}, un'altra blockchain per l'IoT è un progetto fondato in Cina, eredita la stessa struttura del Tangle da IOTA, e dunque ha gli stessi vantaggi e imiti. HDAC \cite{c13} è un'altra blockchain recentemente proposta per l'IoT in Corea, che collabora con il Gruppo Hyundai, e si concentrerà su altri settori specifici dell'IoT come l'autenticazione dei dispositi e le transazioni Machine-to-Machine (M2M).


\section{IoTeX: Panoramica su Architettura e Progetto}

\subsection{Principio di progettazione}
IoTeX mira a diventare la spina dorsale ed il sistema nervoso dedicati all'IoT, scalabile e incentrato sulla privacy. Per raggiungere questo obiettivo e affrontare le sfide citate, il design della nostra architettura è basato sui seguenti princìpi.

\subsubsection{Separazione delle responsabilità}
Connettere direttamente tutti i nodi IoT in una singola blockchain è un sogno che non può essere realizzato. Oltre al fatto che le diverse applicazioni IoT richiedono fondamentalmente diversi gruppi di funzionalità della blockchain, ospitare ogni nodo IoT sulla stessa blockchain la farebbe comunque crescere rapidamente in dimensioni e in richieste computazionali, e alla fine diventerebbe troppo pesante per la maggior parte dei dispositivi IoT. Invece, una separazione delle funzioni assicura che ogni blockchain interagisca con un gruppo specifico di nodi IoT e, allo stesso tempo, interagisca con altre blockchain quando necessario. Analogamente a quanto accade per Internet, dispositivi eterogenei prima formano un gruppo interconnesso, la intranet; intranet più piccole possono ulteriormente formare una intranet più grande, che alla fine si connette alla spina dorsale di internet e tutti i dispositivi possono comunicare tra loro.
La "separazione delle responsabilità" di solito crea un sistema ben bilanciato, per massimizzare sia l'efficienza che la privacy.

\subsubsection{Il Rasoio di Occam}
Diverse blockchain hanno utilizzi e applicazioni diverse, e dovrebbero essere progettate e ottimizzate verso direzioni diverse. Ad esempio: una blockchain dedicata all'inoltro delle transazioni tra le sue subchain non necessita che su di essa vengano eseguiti smart-contract Turing-Completi; un'altra blockchain che colleghi dispositivi appartenenti alla stessa zona di fiducia non dovrebbe preoccuparsi troppo della privacy delle transazioni.

\subsubsection{Convenienza per i dispositivi IoT}
Come già detto, il mondo IoT è pieno di sistemi e nodi eterogenei, più o meno potenti in termini di risorse di calcolo, archiviazione e alimentazione. Dal momento che le operazioni eseguibili dai nodi deboli possono comunque essere facilmente eseguite da quelli forti, le operazioni sulla blockchain dovrebbero essere progettate e ottimizzate per i nodi deboli, ovvero le operazioni dovrebbero essere abbastanza leggere da risparmiare risorse come la potenza di calcolo, lo spazio di archiviazione e l'energia.

\subsection{Architettura: Blockchain in Blockchain}
IoTeX è una rete di molte blockchain disposte gerarchicamente, che possono funzionare in parallelo tra loro pur mantenendo l'interoperabilità. Nel mondo IoTeX, come mostrato nella Figura \ref{fig:fig1}, la blockchain radice (\emph{rootchain}) gestisce molte blockchain indipendenti, o \emph{subchain}. Una subchain si connette e interagisce con quei dispositivi IoT con i quali ha qualcosa in comune, ad esempio che hanno uno scopo funzionale simile, che operano in ambienti simili, o che condividono un livello di fiducia simile. Se una subchain non funziona bene, ad esempio se viene attaccata o si verificano bug del software, la rootchain rimane completamente inalterata. Inoltre, sono supportate transazioni tra le blockchain per trasferire valore e dati dalle subchain alla rootchain, oppure tra una subchain e l'altra attraverso la rootchain.

\begin{figure}[ht]
	\includegraphics[width=\textwidth]{Figura1.png}
	\caption{IoTeX: Blockchain in blockchain, un'architettura composta da una rootchain con subchain}
	\label{fig:fig1}
\end{figure}

La rootchain è una blockchain pubblica accessibile da chiunque, che ha tre obiettivi principali:

\begin{enumerate}
	\item
	      \textbf{Inoltro} di valore e dati tra le subchain, in grado di preservare la privacy, per abilitare l'interoperabilità tra le subchain;
	\item
	      \textbf{Supervisione} delle subchain, ad esempio per penalizzare operatori di subchain confiscandone il deposito;
	\item
	      \textbf{Regolamento e ancoraggio} dei pagamenti e della fiducia per le subchain.
\end{enumerate}
Definiti questi obiettivi, la rootchain si concentra in particolare su scalabilità, robustezza, funzioni di salvaguardia della privacy e capacità di controllare le subchain.
Una subchain, d'altra parte, potrebbe anche essere una blockchain che utilizza la privacy, e dipendere dalla rootchain per l'interazione con altre subchain. Una subchain richiede flessibilità ed estensibilità per adattarsi ai diversi requisiti delle diverse applicazioni IoT. Una subchain sarà molto probabilmente gestita da operatori il cui ruolo è subordinato a un sufficiente deposito di garanzia, ancorato alla rootchain. Opzionalmente, il sistema consente agli operatori di subchain di nominare uno o più operatori terzi che agiscano per loro conto, con o senza vincoli extra. L'operatore agisce come un client leggero all'interno della rootchain, e come un nodo completo nella subchain per generare nuovi blocchi.
Nel complesso, le proprietà di rootchain e subchain sono riassunte in Tabella \ref{table:rootchainandsubchains}

\begin{table}[tp]%
	\caption{Confronto tra Rootchain e Subchain}
	\label{table:rootchainandsubchains}\centering %
	\begin{tabular}{l|l|l}
		\hline
		\textbf{Proprietà}          & \textbf{Rootchain}   & \textbf{Subchain}  \\
		\hline
		Pubblica vs Privata         & Pubblica             & Pubblica o Privata \\
		Scalabile                   & Necessario           & Varia              \\
		Robusta                     & Fortemente Richiesto & Richiesto          \\
		Incentrata sulla Privacy    & Richiesto            & Varia              \\
		Estensibilità               & Non-Turing completa  & Turing completa    \\
		Finalità Istantanea Blocchi & Richiesta            & Richiesta          \\
		\hline
	\end{tabular}
\end{table}

\subsection{Blockchain Radice}
La blockchain radice utilizza il modello basato su UXTO come in Bitcoin \cite{c21} e Monero \cite{c20} per i seguenti motivi:

\begin{itemize}
	\item
	      L'ordinamento delle transazioni diventa banale, non richiede \emph{nonce} o numeri di sequenza, il che pone richieste minime nei metodi di consenso e permette di elaborare le transazioni in parallelo;
	\item
	      Applicando tecniche esistenti di salvaguardia della privacy come la \emph{ring signature} e \texttt{ZKSNARKs}, diventa possibile nascondere il mittente, il destinatario e l'importo della transazione;
\end{itemize}

La blockchain radice è composta da blocchi collegati da hash, e un blocco è costituito da  un'intestazione che lo collega mediante un hash al blocco precedente, oltre che da una lista di transazioni. La rootchain consente principalmente due tipi di transazione: (1) transazioni di base come \texttt{P2PKH, P2SH, Multisig} e così via, e transazioni avanzate che consentono operazioni tra blockchain come \texttt{BondedRegistration, Lock, ReLock, Reorg} ecc.. Le transazioni confermate vengono aggiunte ad un blocco che ha dimensione dinamica, con limite massimo di 8MB. Il nostro sistema di consenso produce un blocco ogni tre secondi come dettagliato nella prossima sezione. La rootchain è progettata per essere non-Turning-Completa, con il supporto di uno script basato su stack e un ricco set di operazioni.

\subsection{Subchain}
IoTeX fornisce di fatto un framework per lo sviluppo e la fornitura di una subchain su misura per applicazioni IoT decentralizzate, incapsulando primitive a basso livello come il protocollo gossip ed il meccanismo di consenso. Il modello di autorizzazione, le specifiche, i parametri e i tipi di transazione della subchain possono essere personalizzati per adattarsi alla propria applicazione.
Le subchain IoTeX utilizzano il modello basato su account, che è più vantaggioso per il tracciamento delle transizioni di stato.
Esistono due tipi di account, similmente a Ethereum: account regolari e contratti. Le transazioni confermate vengono aggiunte al blocco, il quale viene generato con lo stesso meccanismo di consenso usato dalla blockchain radice al fine di ottenere lo stesso livello di finalità, e rendere la comunicazione cross-chain più efficiente. Le subchain possono utilizzare o il token della rootchain, IoTeXtoken, oppure definire il proprio token. I token definiti dagli sviluppatori per le subchain può essere distribuiti mediante vendita pubblica, oppure scambiati sui siti di cambio pubblici.
Le subchain supportano smart contract, che vengono eseguiti su una macchina virtuale leggera ed efficiente. Attualmente stiamo valutando Web Assembly (WASM) \cite{c36}, uno standard Web emergente per la creazione di applicazioni Web ad alte prestazioni. WASM è veloce ed efficiente, può essere reso deterministico, e può essere dotato di sandbox a seguito di piccole modifiche così come tentato dal progetto EOS \cite{c9}, ma vengono esaminate anche altre opzioni. Grazie agli smart contract, i dispositivi IoT collegati alla stessa subchain usano lo stato condiviso in due modi:

\begin{itemize}
	\item Innanzitutto, i dispositivi possono interagire con l'ambiente fisico in base agli stati presenti nella loro subchain: ad es. le lampadine si accendono o spengono autonomamente in base allo stato di un orologio sulla stessa subchain;

	\item D'altra parte, i dispositivi possono aggiornare il proprio stato sulla subchain quando l'ambiente fisico cambia: ad es. il termostato aggiorna la temperatura tramite uno smart contract, in base ai dati provenienti dal proprio sensore di temperatura;
\end{itemize}

\subsection{Comunicazione crosschain}
Ci si aspetta che la comunicazione tra blockchain diverse sarà utilizzata di frequente nelle applicazioni IoT. C'è sempre la necessità per un dispositivo IoT in una subchain di coordinarsi con un altro dispositivo in una diversa subchain. Ancora una volta, limitati dalla bassa potenza di calcolo e dal poco spazio di archiviazione dei dispositivi IoT, siamo motivati a progettare un tipo di comunicazione crosschain che sia veloce, ed economica in termini di risorse.

\subsubsection{Pegging e Finalità dei Blocchi}
Il Pegging è un meccanismo per scalare la rete Bitcoin tramite \emph{"sidechain"}, originariamente
proposto in \cite{c1}. Esso si affida fortemente al \emph{Simplified Payment Verifcation} (SPV) \cite{c21}, e consente ai Bitcoin di "spostarsi" in modo efficiente dalla blockchain Bitcoin a una sidechain
e viceversa. L'idea alla base è semplice: i token vengono inviati ad un indirizzo speciale al fine di essere bloccati sulla blockchain Bitcoin; una volta confermata questa transazione di \texttt{Lock}, si invia la transazione \texttt{Reorg} alla sidechain, includendo il riferimento alla transazione di \texttt{Lock} ed una prova di inclusione (\emph{"Proof of inclusion"}), sotto forma di ramo Merkle. La sidechain usa il SPV per verificare la transazione di \texttt{Reorg} e, se convalidata, crea una quantità di token equivalente e li invia all'indirizzo desiderato sulla sidechain. Ad oggi, il pegging funge da
primitiva per quasi tutti i protocolli di comunicazione cross-blockchain, ad es. Cosmos, Lisk,
Rootstock. Due flussi separati di pegging possono essere facilmente accoppiati insieme per creare il
cosiddetto Pegging a due vie (2WP) che renalizza il trasferimento di token in entrambi i versi.

La finalità dei blocchi rappresenta la garanzia che ogni nuovo blocco generato sia "finale" (cioè \emph{definitivo}), e non possa più essere
modificato. La finalità dei blocchi ha un impatto importante sull'attuazione concreta del pegging:
infatti è necessario aspettare fino a che un blocco sia definitivo (quantomeno con un'alta probabilità) sulla blockchain da cui si invia, prima di poter richiedere la \texttt{Reorg}. La maggior parte delle blockchain pubbliche come Bitcoin non hanno finalità istantanea. In realtà la blockchain ricevente può solo ottenere una sicurezza statistica, dato che man mano che i miner PoW confermano una transazione, aumenta semplicemente la probabilità che essa sia stata accettata nella blockchain. Utilizzare un consenso con finalità risolve questo problema perché la blockchain ricevente ha la garanzia certa già con la conferma di un solo blocco sulla blockchain inviante. Per le applicazioni IoT, il trasferimento di valore e dati tra le blockchain dovrebbe essere veloce, e richiedere poche risorse, e questo impone un meccanismo di consenso con finalità sia sulla rootchain che sulle subchain. Il consenso IoTeX raggiunge la finalità istantanea dei blocchi, come dettagliato nella prossima sezione.

\subsubsection{Protocollo di comunicazione crosschain}
Esaminiamo il protocollo ad alto livello immaginando che un indirizzo di nome \emph{Charlie} sulla
subchain 1 desideri inviare una transazione a un indirizzo di nome \emph{David} sulla subchain 2, e tutte e tre le blockchain usino lo stesso tipo di token, per semplicità senza costi di transazione. Si noti che applicando semplicemente il pegging, saranno necessarie quattro transazioni per effettuare una "remote call" dalla subchain 1 alla subchain 2 attraverso la rootchain, cioè: (1) una transazione di \texttt{Lock} sulla subchain 1; (2) una transazione di \texttt{Reorg} verso la rootchain; (3) un'altra transazione di \texttt{Lock} sulla rootchain; e (4) un'altra transazione di \texttt{Reorg} verso la subchain 2.
Questo processo indica che \emph{David} deve attendere almeno 4 blocchi prima di accettare la "remote call", e i dati che essa trasporta devono essere archiviati su tutte e tre le blockchain, cosa che la rende lenta e computazionalmente costosa. Miriamo a ottimizzare questo processo combinando (2) e (3) in un'unica transazione di \texttt{ReLock}, che non solo accelera l'intero processo ma
evita anche di archiviare i dati nella subchain 1 e nella rootchain. Il nostro protocollo è raffigurato nella Figura \ref{fig:fig2}

\begin{figure}[ht]
	\includegraphics[width=\textwidth]{Figura2.png}
	\caption{Transazioni Cross-Bockchain}
	\label{fig:fig2}
\end{figure}

Il protocollo cross-chain di IoTeX risulta composto dai seguenti passi:

\begin{itemize}
	\item Ogni subchain viene registrata sulla rootchain inviando una transazione chiamata \texttt{BondedRegistration} alla rootchain, che include il nome della subchain, l'ID, la configurazione, il blocco di "genesi", e la nomenclatura degli operatori; questo processo avviene una sola volta;

	\item Quando \emph{Charlie} vuole inviare una transazione a \emph{David}, egli inizia una transazione $\texttt{Lock}(X, H(D), F/T)$ dove $X$ è la quantità di token, $H(D)$ è l'hash dei dati $D$ da allegare, $F/T$ indica gli indirizzi sorgente e destinazione inclusi gli ID per entrambe le subchain;

	\item Una volta che la transazione di \texttt{Lock} è stata inclusa nella blockchain 1, \emph{Charlie} inizia una transazione $\texttt{ReLock(X, H(D), F/T, S, P)}$ con la rootchain includendo $X, H(D), F/T$, alcune statistiche correnti della subchain 1 indicate con $S$ e una "\emph{proof-of-inclusion}" $P$ che comprende i rami Merkle delle intestazioni di blocchi recenti e rami Merkle che provano che la transazione \texttt{Lock} è stata inclusa;

	\item La rootchain valida la transazione \texttt{ReLock}, la accetta includendola nell'ultimo blocco e crea $X$ token bloccandoli in un indirizzo speciale;

	\item Una volta che la transazione \texttt{ReLock} è stata inclusa nella rootchain, \emph{Charlie} invia una transazione $\texttt{Reorg}(X, D, F/T, P')$ sulla rete della rootchain con $X, D, F/T$ ed un'altra \emph{proof-of-inclusion} $P'$ che prova l'inclusione della transazione \texttt{ReLock};

	\item Gli operatori della subchain 2 si accorgono della transazione \texttt{Reorg}, dunque validano e creano la stessa quantità di token sulla subchain 2, inviandoli all'indirizzo di \emph{David} con associati i dati $D$.

\end{itemize}

\subsubsection{Condivisione della larghezza di banda della Rootchain}
Una possibile proeccupazione che riguarda la comunicazione crosschain, è che alcune subchain malevoli possano generare \emph{spam} sulla rootchain o su un'altra subchain trasmettendo un'enorme quantità di transazioni crosschain, ed esaurendo così la capacità dell'altra blockchain. Un modo di attenuare il problema è di richiedere ad ogni subchain di appaltare una propria quota di transazioni, e di limitare la frequenza delle transazioni provenienti da una subchain se la sua quota si esaurisce.

\begin{figure}[ht]
	\includegraphics[width=\textwidth]{Figura3.png}
	\caption{Modello della Largezza di Banda per Condividere la Capacità della Rootchain}
	\label{fig:fig3}
\end{figure}

Si potrebbe definire una quota basandosi sullo spazio all'interno di un blocco. Supponiamo che la dimensione massima di un blocco sia di 8MB, e che 4MB siano riservati per le normali transazioni all'interno della rootchain, mentre 4MB siano riservati per tutte le transazioni cross-blockchain, ulteriormente suddivisi in, diciamo 4096 parti di quota, con ogni parte di 1KB. Una subchain richiede l'allocazione $n$ parti di quota (con un limite massimo prefissato) per gli usi desiderati, versando una cauzione proporzionale ad $n$. Ad ogni ciclo, solo $n$KB possono essere occupati all'interno di ciascun blocco per le transazioni provenienti da quella subchain e per ognuna di esse viene scalata una "commissione di banda" dal deposito (per premiare i miner che lavorano per applicare questa regola); le transazioni rimanenti vengono accodate e infine scartate oltrepassato un tempo massimo. L'allocazione delle quote può essere dinamica nel senso che può subire cambiamenti quando la rootchain cresce, come mostrato in Figura \ref{fig:fig3}. Se una subchain inviasse spam alle altre, consumerebbe il proprio deposito molto velocemente ed alla fine perderebbe la propria quota di banda. D'altra parte, se una subchain versasse un grosso deposito col solo scopo di occupare una gran parte della larghezza di banda senza in realtà utilizzarla, la rootchain avrebbe un meccanismo per risarcire parte del deposito secondo il rapporto tra il numero medio di transazioni per blocco e la porzione di banda riservata: questo stabilizzerebbe la quantità di banda richiesta dalle subchain su valori vicini a quella effettivamente utilizzata.

\section{Transazioni con Conservazione della Privacy Integrate}
La privacy fornita in modo nativo da Bitcoin ed Ethereum è limitata alla pseudonimia. I dettagli della transazione non sono riservati. L'importo della transazione ed i beni trasferiti, i loro metadati e le relazioni con altre transazioni possono essere banalmente verificate da chiunque. In effetti, in questo contesto, ci sono tre aspetti della privacy: la privacy del mittente, la privacy del destinatario e la privacy dei dettagli della transazione. Vari schemi crittografici possono essere applicati per affrontarli, come mostrato nella Tabella \ref{table:PrivacyPreservingTechniques}.

\begin{table}[tp]%
	\caption{Tecniche di Conservazione della Privacy per le Blockchain}
	\label{table:PrivacyPreservingTechniques}\centering %
	\begin{tabular}{l|p{2cm}c|p{2.5cm}|p{2cm}}
		\hline
		Tecnica              & Nasconde il Mittente & Nasconde il destinatario & Nasconde l'Importo \\
		\hline
		Stealth Address      & N                    & Y                         & N                  \\
		Pedersen Commitments & N                    & N                         & Y                  \\
		Ring Signatures      & Y                    & N                         & N                  \\
		zk-SNARKs            & Y                    & N                         & Y                  \\
		\hline
	\end{tabular}
\end{table}

IoTeX integra lo stealth address per la privacy del destinatario, la ring signature per la privacy del mittente ed il Pedersen Commitments per nascondere l'importo della transazione, con le seguenti innovazioni e miglioramenti:

\begin{itemize}
	\item Uno schema di indirizzo invisibile leggero è progettato per esentare i destinatari dalla scansione
	      l'intera blockchain per essere a conoscenza delle transazioni in entrata;

	\item La ring signature è ottimizzata per ridurne le dimensioni con una configurazione distribuita di tipo trusted.
\end{itemize}


\subsection{Nascondere il destinatario della transazione mediante Codice di Pagamento Inoltrabile}

\subsubsection{Stealth Address}
La tecnica dello stealth address nasce dal protocollo Cryptonote \cite{c28}, che risolve il problema del destinatario utilizzando un protocollo di scambio di chiavi Diffie-Hellman a "mezzo giro". Supponendo che Bob voglia nascondere il fatto che riceverà dei token da Alice, ecco come funzionerebbe:

\begin{enumerate}
	\item Bob crea due coppie di chiavi private e pubbliche, indicate come $(a, A)$ e $(b, B)$, dove $A = a\cdot G$ e $B = b\cdot G$, dove $G$ è il punto base su una curva ellittica.

	\item Bob divulga le chiavi pubbliche $(A, B)$, note come il suo stealth address;

	\item Alice calcola e invia i token a $P = H (rA)\cdot G + B$ usando una funzione hash $H$, un numero casuale e grande $r$  e lo stealth address di Bob $B$. Questa transazione viene trasmessa insieme a $R = r\cdot G$;
	\item  Bob monitora tutte le transazioni, calcola $P' = (H(aR) + b)\cdot G$ (poiché egli conosce $a$, $b$, $R$ e $G$) con la speranza che $P'$ sia uguale a $P$. Se $P' = P$, Bob potrà spendere i token inviati a $P'$ con la chiave privata $H(aR) + b$.
\end{enumerate}

Un ovvio inconveniente dello stealth address è che il destinatario deve scansionare tutte le transazioni della rete (il che non è l'ideale nel mondo IoT) oppure deve basarsi sull'assistenza di un full node di fiducia (il che a un certo livello compromette la privacy).

\subsubsection{Codice di pagamento}
Il codice di pagamento è stato ideato per risolvere l'inconveniente di cui sopra, relativo allo stealth address sacrificando in parte la privacy. L'idea è che Alice notifichi a Bob un codice di pagamento in un modo confidenziale, e Bob monitori solo le transazioni verso gli indirizzi derivanti da quel codice. Pertanto, questa proposta presenta due flussi: quello della notifica, che rappresenta una configurazione una-tantum tra certe due parti, e quello di invio, che può accadere più volte tra queste due parti.

Supponendo che Alice abbia una coppia di chiavi pubblica-privata principali ($mpub_{Alice}, mpri_{Alice}$) dove $mpub_{Alice} = mpri_{Alice}\cdot G$, ed una coppia di chiavi pubbliche-private di portafoglio ($wpub_{Alice}, wpri_{Alice}$) dove $wpub_{Alice} = wpri_{Alice}\cdot G$; che Bob abbia una coppia di chiavi pubblica-privata principali $(mpub_{Bob}; mpri_{Bob})$ dove $mpub_{Bob} = mpri_{Bob}\cdot G$, la notifica una tantum funziona come descritto di seguito:

\begin{enumerate}
    \item Bob deriva $B_0 = b_0\cdot G = (mpri_{Bob} + Hash(0, seed, metadata))\cdot G$, lo converte in un indirizzo di notifica $\texttt{addr}(B_0)$, lo pubblica e si mette in ascolto su di esso

    \item  Alice sceglie un chain code $cc$ a caso; $(mpub_{Alice}||cc)$ è il codice di pagamento per Alice;

    \item  Alice calcola un codice segreto condiviso $S = wpri_{Alice}\cdot B_0$ ed invia il codice di  pagamento mascherato $P' = (mpub_{Alice}||cc)\oplus HMAC512(xofS)$ ad $\texttt{addr}(B_0)$;

    \item Al ricevimento, Bob ottiene $wpub_{Alice}$ e recupera $S = wpub_{Alice}\cdot b_0$, scopre $P'$ per ottenere $(mpub_{Alice}||cc)$.
\end{enumerate}

Una volta che il flusso di notifica è completo, Alice e Bob stabiliscono un canale privato unidirezionale per l'invio di token. Il primo invio funziona come descritto di seguito:

\begin{enumerate}
    \item Alice deriva un nuovo indirizzo dal suo codice di pagamento (che è già condiviso von Bob) da $A_0 = a_0\cdot G = mpub_{Alice} + Hash(0, seed, metadata)\cdot G$;

    \item Alice seleziona la successiva chiave pubblica non ancora utilizzata, derivata da $B_0$. Si noti che $B_0$ è la chiave pubblica inutilizzata per il primo round.

   \item Alice calcola il nuovo codice segreto condiviso $S_0 = a_0\cdot B_0$ e calcola la chiave pubblica temporanea utilizzata per inviare la transazione per cui si verifica $B'_0 = B_0 + SHA256(S_0)\cdot G$

   \item Bob potrebbe derivare $A_0$ in modo non interattivo poiché conosce il codice di pagamento di Alice, e ascolta solo l'indirizzo derivato da $B'_0 = B_0 + SHA256(S_0)\cdot G$ ed $S_0 = A_0\cdot b_0$.

   \item Alla ricezione, Bob può spendere i token con la chiave privata $b_0 + SHA256(S_0)$.

\end{enumerate}

I flussi successivi funzionano in maniera analoga.
Bob non ha bisogno di monitorare la rete o affidarsi a un full node per eseguire la scansione di tutte le transazioni. La transazione di notifica fa trapelare l'intenzione di Alice di inviare qualcosa a
Bob, ma l'effettivo "invio di qualcosa" è nascosto a tutti gli altri.

\subsubsection{Codice di Pagamento Inoltrabile}
Per ridurre ulteriormente la perdita di privacy, abbiamo progettato il codice di pagamento inoltrabile basandoci sulla codice di pagamento originariamente proposto. Mentre il flusso di invio rimane lo stesso, abbiamo migliorato il flusso di notifica per consentire ad Alice di condividere segretamente il suo codice di pagamento con Charlie senza utilizzare la transazione di notifica, assumendo che Alice e Bob abbiano un canale privato unidirezionale, e Bob e Charlie abbiano un altro
canale privato unidirezionale. Per ottenere ciò, sfruttiamo i contratti Hashed Timelock (HTLC - \emph{Hashed TimeLock Contracts}), i quali richiedono che il destinatario di un pagamento confermi la ricezione di un pagamento prima di una deadline generando una \emph{proof of payment} crittografata, oppure rinunci alla possibilità di reclamare il pagamento, restituendolo al mittente.


Supponendo che Charlie abbia una coppia di chiavi pubblica-privata $(mpub_{Charlie}, mpri_{Charlie})$ dove $mpub_{Charlie} = mpri_{Charlie}\cdot G$. Il flusso di notifica migliorato funziona come illustrato di seguito:

\begin{enumerate}
    \item Charlie deriva $C_0 = c_0\cdot G = (mpri_{Charlie} + Hash(0, seed, metadata))\cdot G$, lo converte in un indirizzo di notifica $\texttt{addr}(C0)$, lo pubblica. Si noti che $C_0$ è pubblicato per Alice al fine calcolare il codice segreto condiviso, ma non per ricevere     transazioni;

    \item Alice genera il suo codice di pagamento $(mpub_{Alice}||cc)$ nello stesso modo;

    \item Alice calcola un codice segreto condiviso $S = wpri_{Alice}\cdot C_0$ e invia il codice di pagamento mascherato $P' = (mpub_{Alice}||cc) \oplus HMAC512(xofS)$ con $X$ token come incentivo e $HTLC(Hash^2(cc))$ a Bob usando il loro canale privato unidirezionale, dove $HTLC$, come parte dello script di blocco o riscatto, afferma che i token diventano
    spendibili se viene fornita la pre-immagine di $Hash^2(v)$, ovvero $Hash (cc)$;

    \item Bob, incentivato dai token inviati da Alice, invia $P'$, $Y$, $Y < X$ token ed $HTLC(Hash^2(v))$ a Charlie usando il loro canale privato unidirezionale;

    \item Charlie, dopo aver ricevuto la transazione di Bob, calcola $S = wpub_{Alice}\cdot c_0$ per scoprire il codice di pagamento di Alice, e spendere la transazione rivelando $Hash(cc)$, che rende spendibile la transazione da Alice a Bob, che premia Bob.
\end{enumerate}

Una volta che questo flusso viene completato, Alice e Charlie stabiliscono un canale privato unidirezionale per l'invio di token. È interessante notare che il tragitto della transazione di Alice potrebbe consistere di più salti.

I nostri codici di pagamento inoltrabili offrono una privacy maggiore in termini di occultamento dell'intenzione di "inviare qualcosa" sulla blockchain, sfruttando i canali privati esistenti, e senza aggiungere alcun overhead di elaborazione o di archiviazione ai nodi. Inoltre, sebbene progettati per scenari IoT, i codici di pagamento inoltrabili risultano utilizzabili per la maggior parte delle blockchain come Bitcoin.

\section{Consenso Veloce con Finalità Istantanea}

\subsection{Contesto}

\subsubsection{Proof of Work}
Il Proof of Work (PoW) è l'infrastruttura per raggiungere il consenso globale nella maggior parte delle blockchain, incluso Bitcoin ed Ethereum. Il PoW rende computazionalmente difficile costruire un blocco valido e collegarlo alla blockchain. Più lunga diventa la blockchain, più difficile diventa annullare qualunque transazione precedentemente archiviata dalla blockchain. Per manipolare la blockchain, un attaccante deve possedere il 51 percento dell'intera potenza di calcolo in una rete blockchain basata su PoW.
Sebbente il PoW fornisca una soluzione elegante per il consenso globale in una blockchain distribuita di grandi dimensioni, esso ha alcuni svantaggi intrinseci. Il costo computazionale complessivo per mantenere il consenso globale è pari allo stesso costo dell'attacco del 51 percento. Questo significa che anche se la maggioranza dei partecipanti nella blockchain sono onesti, essi devono comunque utilizzare molta elettricità per sostenere la blockchain, il che non si addice per l'ambiente delle reti IoT che in genere predilige l'efficienza energetica. Inoltre, a livello dei singoli dispositivi, calcolare il PoW in genere costa molti cicli di CPU e utilizzo di memoria, il che pone requisiti difficili per la realizzazione dell'hardware ed i costi dei dispositivi IoT integrati. Ultimo ma non ultimo, il PoW non fornisce finalità istantanea che è una proprietà fondamentale richiesta per realizzare una comunicazione cross-chain efficiente.

\subsubsection{Proof of Stake}
Il Proof of Stake (PoS) è stato proposto come un'alternativa effciente al PoW per il raggiungimento del consenso nelle blockchain, e mira ad evitare i suddetti problemi del PoW. L'idea di base del PoS è che un insieme di nodi scelti a caso votino il blocco successivo, e i loro voti siano ponderati in base alla dimensione dei loro depositi (ovvero gli stake). Se alcuni nodi si comportano male, essi rischiano di perdere il loro deposito. In questo modo, senza il PoW con i suoi pesanti requisiti computazionali, la blockchain può funzionare molto più effcientemente, e può raggiungere una stabilità economica: maggiore è il deposito di un partecipante, maggiore è l'incentivo per quel nodo a mantenere il consenso globale, e meno probabile è che il nodo si comporti male. Esistono un paio di progetti e implementazioni pubbliche del PoS, come ad esempio Tendermint \cite{c32} che è stato adottato da molte applicazioni \cite{c33}.

\subsubsection{Delegated Proof of Stake (DPoS)}
Il Delegated Proof of Stake (DPoS) migliora l'idea del PoS poiché consente ai partecipanti di scegliere alcuni delegati per rappresentare le loro porzioni di deposito nella rete. Ad esempio, Alice può inviare un messaggio alla rete per garantire a Bob la possibilità di rappresentare il suo deposito e votare a proprio nome. Il DPos offre diversi vantaggi per le nostre applicazioni IoT:

\begin{itemize}
	\item I nodi con portafogli piccoli possono mettere insieme i loro depositi per avere più possibilità insieme di partecipare alla proposta e votazione del prossimo blocco, e in seguito condividere la ricompensa.

	\item I nodi con risorse limitate possono scegliere i propri delegati, e quindi non tutti i nodi necessitano di rimanere online per contribuire al consenso.

	\item I delegati possono essere quei nodi con alimentazione e condizioni di rete affidabili, e possono anche essere scelti in modo dinamico e casuale, avendo quindi una maggiore disponibilità generale per far sì che la rete raggiunga il consenso.
\end{itemize}

Criptovalute tipiche che utilizzano il DPoS includono EOS \cite{c9} e Lisk \cite{c18}.

\subsubsection{Practical Byzantine Fault Tolerance}
La Practical Byzantine Fault Tolerance (PBFT) è stata proposta da Castro e Liskov \cite{c7} nel 1999 come algoritmo efficiente e resistente agli attacchi per raggiungere accordi in una rete asincrona distribuita. Prevediamo di utilizzare PBFT per l'algoritmo di votazione sottostante del nostro meccanismo di consenso DPoS, perché è un algoritmo conciso e ben studiato che fornisce finalità rapida, il che è di fondamentale importanza per la costruzione di una
blockchain efficiente e stabile. Come dimostrato nell'articolo originale di Castro e Liskov, il PBFT o offre sia disponibilità che sicurezza se un terzo dei nodi della rete al massimo sono difettosi o malevoli, e il costo di rete del PBFT è molto basso, ad esempio circa il 3 percento rispetto a un sistema di rete non replicato. Criptovalute tipiche basate su PBFT includono Stellar \cite{c30} e Zilliaq \cite{c38}.

\subsection{Delegated Proof of Stake Randomizzato (R-DPoS)}
Per ottenere un meccanismo di consenso rapido ed efficiente con finalità istantanea del blocco nel contesto dell'IoT, combiniamo i concetti di DPoS, PBFT e Verifiable Random Functions (VRF). VRF è stato introdotto per la prima volta da Micali et al. in \cite{c19} e rappresenta una famiglia di funzioni che possono produrre prove verificabili pubblicamente per la correttezza dei loro output casuali. Ad alto livello, il R-DPoS proposto ha quattro fasi: \emph{elezione dei candidati}, \emph{formazione della commissione }, \emph{proposta del blocco} e \emph{finalizzazione del blocco}.

\begin{figure}[ht]
	\centering
	\includegraphics[width=0.5\textwidth]{Figura6}
	\label{fig:figure6}
	\caption{Randomized Delegated Proof of Stake (R-DPoS)}
\end{figure}

\subsubsection{Elezione dei candidati}
Tutti i nodi della rete IoTeX possono partecipare in questa fase votando per i candidati alla commissione. Per incentivare i nodi a votare, il sistema assicura che i delegati condividano le ricompense forgiate con i loro elettori. I candidati formano un insieme di almeno 97 delegati; questo numero aumenterà in futuro per evitare ulteriormente la centralizzazione della potenza di estrazione. Una volta selezionati i candidati, questi saranno fissati per un'epoca,
che consiste di 47 iterazioni.

\subsubsection{Formazione della commissione}
In ogni iterazione, viene selezionata una commissione di 11 elementi scelti a caso dal pool di candidati mediante VRF per la creazione dei blocchi nei successivi 11 round. L'idea è di usare l'hash del blocco dell'ultima iterazione e la chiave privata del nodo come input del VRF per produrre un output Booleano che indica se esso è stato selezionato come membro della commissione, una priorità che indica il suo livello per proporre un blocco, e una prova che indica i suoi requisiti per poter proporre il blocco in un certo round. L'uso di VRF è importante in quanto fornisce un modo non interattivo per ordinare tutti i delegati per proporre blocchi in modo equo e in sicurezza. A tal fine, usiamo il VRF efficiente come quello utilizzato in Algorand \cite{c12}.

\subsubsection{Proposta del blocco}
In ogni round (che inizia all'incirca ogni 3 secondi), ogni nodo della commissione propone un nuovo blocco e lo trasmette alla rete, insieme con la priorità e la prova (forniti dal VRF). Solo il blocco proposto da un nodo della commissione con la priorità più alta e che non è stato già proposto nella stessa iterazione viene considerato dagli altri nodi, e viene chiamato blocco candidato.

\subsubsection{Finalizzazione del blocco}
Nello stesso round, tutti gli altri nodi utilizzano PBFT per votare a favore o contro il blocco candidato. Se più dei 2/3 dei nodi della commissione concordano sulla validità del blocco candidato, esso vieen finalizzato ed aggiunto alla blockchain da tutti gli utenti della rete. Dopo di ciò, i passi \emph{proposta del blocco} e \emph{finalizzazione del blocco} vengono nuovamente eseguiti nel round successivo; se l'iterazione corrente è terminata, sarà formata un'altra commissione a caso, prima che \emph{proposta del blocco} e \emph{finalizzazione del blocco} vengano eseguiti.

\subsection{Creazione di checkpoint periodici per i client leggeri}
Nelle reti IoT, ci aspettiamo che molti dispositivi siano dei client leggeri, che rappresentano i partecipanti della blockchain che non registrano la cronologia completa delle transazioni localmente. Considerando l'overhead di archiviazione della blockchain completa, ad es. oltre 100 GB per Bitcoin \cite{c4}, molti dispositivi embedded IoT a basso costo potrebbero non avere la capacità di scaricare la blockchain completa. Tuttavia, questi client leggeri hanno ancora la capacità di verificare rapidamente la correttezza della blockchain e di interagire con essA: l'idea è inclusa nel Whitepaper originale del Bitcoin di Satoshi \cite{c21}.
Tuttavia, l'utilizzo del PoS anziché del PoW ha uno svantaggio per i client leggeri. Quando si verifica la correttezza di una blockchain basata sul PoS, i client devono scaricare un elenco di chiavi pubbliche e firme per i proponenti di blocco e gli elettori, ed i gruppi di proponenti di blocco
e gli elettori possono cambiare per ogni blocco. Quindi, quando i client leggeri tornano online dopo essere stati offline per un po', i client potrebbero aver bisogno di scaricare un gran numero di chiavi pubbliche e firme, e quindi verificarle tutte. Per mitigare questo problema di performance, Vitalik, l'inventore di Ethereum, ha proposto di creare dei checkpoint periodici sulla blockchain, chiamate \emph{epoche} \cite{c6}, ad esempio ogni 50 blocchi. Ogni checkpoint
può essere verificato basandosi sul checkpoint precedente, in modo tale che i client leggeri possano sincronizzarsi con l'intera blockchain molto più velocemente.



\section{Token sulla rete IoTeX}
Il token digitale protetto crittograficamente nativo della rete IoTeX (IOTX) è un componente principale dell'ecosistema della rete IoTeX stessa, ed è progettato per essere utilizzato esclusivamente sulla rete. Prima del lancio della mainnet IoTeX, esso esisterà come token compatibile ERC20 sulla blockchain Ethereum, che sarà trasformato iu un token sulla mainnet IoTeX quando questa sarà avviata.

IOTX è richiesto quale "cripto-carburante" virtuale per l'utilizzo di determinate funzioni progettate sulla rete IoTeX (come l'esecuzione di transazioni e l'esecuzione di applicazioni distribuite sulla rete IoTeX), fornendo gli incentivi economici che saranno consumati per incentivare i partecipanti a contribuire a sostenere l'ecosistema sulla Rete IoTeX. Sono necessarie risorse computazionali per l'esecuzione di varie applicazioni, e per eseguire transazioni sulla rete IoTeX, così come la convalida e la verifica di ulteriori blocchi/informazioni sulla blockchain, quindi i fornitori di questi servizi/risorse hanno bisogno di incentivi economici per la fornitura di tali risorse (come ad es. avviene per il "mining" sulla rete IoTeX) per mantenere l'integrità della rete, e IOTX sarà utilizzato come unità di scambio per quantificare e pagare i costi del consumo di risorse computazionali. IOTX sarà "estraibile" per 50 anni, e le ricompense per l'attività di estrazione si ridurranno nel tempo sulla base di un modello di riduzione a gradiente lineare.

IOTX è una parte integrante e indispensabile della rete IoTeX, perché in assenza di IOTX, non ci sarebbe alcuna unità di scambio comune per pagare tali costi, rendendo così insostenibile l'ecosistema basato sulla rete IoTeX.

IOTX è un token di utilità funzionale non rimborsabile che verrà utilizzato come unità di scambio tra i partecipanti sulla rete IoTeX. L'obiettivo di introdurre IOTX è di fornire una modalità di pagamento e di accordo conveniente e sicura tra i partecipanti che interagiscono all'interno dell'ecosistema sulla rete IoTeX. IOTX non rappresenta in alcun modo una quota azionaria, partecipazione, diritto, titolo o interesse in IoTeX Foundation Ltd. (la \textbf{Fondazione}), nei suoi affiliates o in qualsiasi altra società, impresa o iniziativa, né IOTX autorizzerà i titolari di token ad alcuna promessa di commissioni, entrate, profitti o rendimenti di investimento e non sono intesi a costituire titoli a Singapore o in qualsiasi giurisdizione pertinente. IOTX può essere utilizzato solo sulla rete IoTeX, e possedere IOTX non dà diritti, espliciti o impliciti, diversi dal diritto di utilizzare IOTX come mezzo per consentire l'utilizzo e l'interazione con la rete IoTeX.

In particolare, IOTX:

\begin{enumerate}[label=(\alph*)]
    \item non è rimborsabile e non può essere scambiato con denaro contante (o il suo valore equivalente in qualsiasi altra valuta virtuale) o qualsiasi obbligo di pagamento da parte della Fondazione o qualsiasi affiliato;

    \item non rappresenta o conferisce al titolare del token alcun diritto in qualsiasi forma relativamente alla Fondazione (o a uno qualsiasi dei suoi affiliati) o alle sue entrate o attività, incluso, ma non limitato a, alcun diritto di ricevere entrate, azioni, diritto o quota su proprietà future, quota o titolo, voto, distribuzione, rimborso, liquidazione, proprietà (comprese tutte le forme di proprietà intellettuale) o altri aspetti finanziari
    o diritti legali o diritti equivalenti, o diritti di proprietà intellettuale o qualsiasi altra forma di partecipazione in relazione alla rete IoTeX, alla Fondazione, al Distributore e/o loro fornitori di servizi;

    \item non è inteso a rappresentare denaro (compreso denaro elettronico), titoli, beni, obbligazioni, strumenti di debito o qualsiasi altro tipo di strumento finanziario o investimento;

    \item non rappresenta un prestito alla Fondazione o a uno qualunque dei suoi affiliati, non è destinato a rappresentare un debito dovuto dalla Fondazione o da qualcuno dei suoi affiliati, e non esiste alcuna attesa di profitto; e

    \item non fornisce al titolare del token alcuna proprietà o altro interesse nei confronti della Fondazione o alcuno dei suoi affiliati.
\end{enumerate}


\section{Ecosistemi Basati su IoTeX}

La blockchain IoTeX supporta una varietà di ecosistemi IoT: shared economy, smart home, veicoli autonomi, supply chain, ecc. Diversi tipi di sviluppatori possono sfruttare IoTeX in modi diversi. Gli sviluppatori supportati da IoTeX includono produttori di hardware IoT, sviluppatori di sistemi di controllo dei dispositivi IoT, sviluppatori di app per smart home, produttori di dispositivi per la shared economy, integratori di dati della supply chain, venditori di data-crowdsourcing, sviluppatori di automobili a guida autonoma, ecc. Questa sezione descrive alcuni ecosistemi basati su IoTeX.

\subsection{Shared Economy}
Negli ultimi anni, molte aziende si sono concentrate sulla shared economy, dalla condivisione dei viaggi realizzata da Uber/Lyft/Didi, alla condivisione di alloggi di Airbnb, alla condivisione di biciclette di Mobike/ofo, alla condivisione di piccoli oggetti come power bank, ombrelli, ecc... Tutti migliorano la vita delle persone, anche se alcuni business ne ricevono un danno. Discutere questi modelli di business non è oggetto di questo documento: qui ci concentriamo principalmente sulla loro architettura tecnologica. Tra tutte le economie condivise, la condivisione degli spostamenti in auto è l'unico che al momento non può fare a meno del lavoro dell'uomo, ovvero del conducente: non è un'economia basata su IoT. Tuttavia, in futuro, quando la tecnologia delle auto a guida autonoma sarà matura e diffusa, la condivisione dei viaggi sarà anch'essa basato su IoT.

Le economie condivise basate su IoT hanno alcune somiglianze tra loro: tutte richiedono un canone di locazione ed una qualche forma di "serratura" che può essere "sbloccata" da una cauzione. È assolutamente possibile ed anche efficiente basare l'intero processo di condivisione e Attualmente, queste economie sono basate su un cloud centralizzato, e questo porta a vari inconvenienti:

\begin{enumerate}
    \item Un ingente deposito cauzionale è detenuto da una società che potrebbe non essere affidabile. Di recente ci sono stati molti casi in cui l'azienda che gestiva un servizio di bici condivise in Cina non è stata in grado di restituire i depositi ai propri clienti;

    \item Le economie condivise non sono interamente gestite dalla comunità. Molti oggetti condivisi sono di proprietà di un'azienda e ciò ha causato uno spreco di risorse. Prendiamo le biciclette condivise come esempio: quando le aziende di condivisione bici chiudono l'attività, quelle bici vengono eliminate.

    \item A causa della loro natura centralizzata, i dati dell'utente saranno archiviati e controllati da una società. Ci sono rischi che il server o il client delle società possano essere violati al fine di ottenere i dati degli utenti.
\end{enumerate}

IoTeX, come infrastruttura, potrebbe essere utilizzato per creare queste applicazioni senza i problemi di cui sopra, e rendere le economie condivise decentralizzate e più efficienti. In concreto, un'economia condivisa basata su IoTeX offre i seguenti vantaggi:

\begin{figure}[ht]
    \centering
    \includegraphics[width=\textwidth]{Figura7}
    \caption{Shared Economy basata su IoTeX}
    \label{fig:Figura7}
\end{figure}

\begin{enumerate}
    \item Il deposito cauzionale è interamente regolato da uno smart contract. Poiché nessuno trattiene i soldi, la restituzione del deposito è sempre garantito. Gli utenti non sono obbligati a fidarsi di una compagnia per utilizzare il servizio.

    \item Ogni oggetto condiviso realizza il suo valore e la sua missione autonomamente. Nell'ecosistema, non importa chi possiede gli oggetti in esso condivisi. Chiunque può possederli e contribuire all'ecosistema. L'economia condivisa può essere gestita dalla comunità. Di conseguenza, le aziende possono svolgere il ruolo di realizzare/manutenere i dispositivi IoT di blocco (la "serratura") e gestire la community. È un modello di business molto più leggero che le aziende possono espandere velocemente per servire più persone.

    \item Ancora una volta, gli utenti non devono fidarsi dell'azienda per mantenere i propri dati. I dati sono mantenuti nella blockchain IoTeX, con anche la protezione della privacy degli utenti.
\end{enumerate}

La Figura \ref{fig:Figura7} descrive come funziona la shared economy basata sulla blockchain IoTeX.

\subsection{Smart Home}

Nel mercato corrente della smart home, molti produttori di dispositivi IoT continuano a utilizzare tecnologie obsolete per sviluppare i loro prodotti. Hanno bisogno di una grande quantità di lavoro di sviluppo per i loro cloud. Il costo di sviluppo e manutenzione è elevato, e le prestazioni sono basse a causa del tragitto di andata/ritorno richiesto verso/dal cloud. Distribuendo i loro prodotti sulla blockchain IoTeX, il produttore ridurrà in gran parte i costi operativi di ingegnerizzazione e di cloud computing e, allo stesso tempo, aumenterà notevolmente le prestazioni dei loro dispositivi. Nel semplice esempio di una lampadina intelligente, con la tecnologia cloud, occorrono due tragitti dei dati dall'istante in cui l'utente comanda di cambiare lo stato della lampadina. I produttori di hardware non sono esperti di cloud, così spesso il loro servizio non è ottimale: la comunicazione, tra l'andata e il ritorno, può durare da uno a tre secondi. Ciò li costringe a utilizzare i servizi cloud di grandi aziende IT, e ci sono due aspetti negativi dell'utilizzo di questi servizi cloud:

\begin{enumerate}
    \item I produttori di hardware IoT non possono controllare completamente la disponibilità dei servizi cloud.

    \item Devono di pagare continuamente per il servizio cloud, a fronte di un incasso una-tantum per la vendita dei loro dispositivi IoT.

    \item Ci sono rischi di hacking del loro cloud: in caso di hacking lato client o intranet i dati degli utenti verrebbero trafugati o addirittura si potrebbero creare problemi di sicurezza domestica.
\end{enumerate}

Al contrario, la blockchain IoTeX gestisce i dispositivi localmente, e interagisce con la blockchain pubblica su internet solo quando necessario. La blockchain pubblica è gestita dalla comunità. Non ci sono costi di manutenzione per i produttori di IoT. La blockchain di IoTeX dispone di protezione della privacy, il che può impedire la rivelazione di dati sensibili oppure che l'unità di controllo venga hackerata, anche nel caso che la rete intranet dell'utente non sia sicura.

\begin{figure}
    \centering
    \includegraphics[width=\textwidth]{Figura8}
    \caption{Smart Home basata su IoTeX}
    \label{fig:Figura8}
\end{figure}

Oltre a consentire ai produttori di IoT di implementare i loro dispositivi sulla propria blockchain, IoTeX collaborerà con i produttori di chip IoT per sviluppare chip abilitati all'uso della blockchain IoTeX, per accelerare il ciclo di progettazione e produzione di dispositivi IoT. I produttori di dispositivi IoT potranno semplicemente integrare il chip per fare in modo che i loro dispositivi siano immediatamente supportati dalla blockchain IoTeX.


\subsection{Gestione delle identità}
Il mondo in crescita dell'IoT ha avuto un impatto su come la gestione delle identità e degli accessi (\emph{Identity and Access Management} - IAM) funzionerà. In termini di identità delle cose, lo IAM deve essere in grado di gestire il sistema utente-dispositivo, dispositivo-dispositivo, e/o dispositivo-servizio. Un modo semplice per la gestione dell'identità è quello di considerare la blockchain IoTeX come un sistema PKI decentralizzato (grazie alla sua immutabilità), in cui a ciascuna entità viene rilasciata un'identità crittografica sotto forma di certifcato TLS con la chiave privata corrispondente. Questo certificato, che tendenzialmente è di breve durata, viene firmato dal certificato di lunga durata integrato nel dispositivo, e poi pubblicato sulla blockchain IoTeX (rootchain o subchain). I nodi ed altre entità possono accedere e fidarsi del certificato di breve durata ancorato alla
blockchain, e i dispositivi possono quindi autenticarsi quando vanno online, garantendo sicurezza di comunicazione con gli altri dispositivi, servizi e utenti, e dimostrare la loro integrità.

Inoltre, è possibile organizzare gerarchicamente i certificati di lunga durata integrati nei dispositivi, come per la PKI convenzionale, in cui i dispositivi genitore possono firmare i certificati dei dispositivi figli. Grazie alla gerarchia, diventa possibile la revoca e la rotazione dei certificati. Ad esempio, se un dispositivo viene compromesso, il suo dispositivo genitore o anche il dispositivo nonno potrebbero firmare un comando di revoca e inviarlo alla blockchain dove quest'ultimo invalida il certificato del dispositivo compromesso.

\section{Lavori di ricerca futuri}
Alcune direzioni di ricerca già in corso ed altre future per migliorare IoTeX sono indicate di seguito.

\paragraph{Potenza di calcolo per il mantenimento della privacy}
Ci sono diverse aree in questa direzione che stiamo esplorando attivamente:

\begin{itemize}
    \item Come mantenere gli stati confidenziali sulla blockchain, in modo che possano essere utilizzati per il calcolo da un certo gruppo di nodi;

\item Smart contract in grado di preservare la privacy, dove lo smart contract può essere eseguito nonostante la sua business logic sia protetta dalla crittografia. Mentre la crittografia completamente omofobica \cite{c26} e gli schemi di offuscamento indistinguibile \cite{c11} rappresentano il "Santo Graal" solo in teoria, proposte pratiche come Hawk \cite{c17} sono promettenti per il prossimo futuro;

\item Ulteriore risuzione dei requisiti computazionali e dello spazio di archiviazione necessari per le tecniche di preservazione della privacy che IoTeX sta attualmente utilizzando;

\item La versione quantum-safe delle tecniche di conservazione della privacy che IoTeX utilizza attualmente, ad esempio una ring signature quantum-safe.
\end{itemize}


\paragraph{Pruning e Trasferimento degli Stati}
Stiamo valutando diversi modi per sfoltire in maniera sicura gli stati memorizzati nelle subchain, per ridurre l'ingombro di storage dal momento che molti dispositivi IoT dispongono di spazio di archiviazione limitato. La compressione di blocchi e transazioni risulta sicuramente una soluzione comoda. Inoltre, anche trasferire gli stati da una subchain alla rootchain (dal momento che quest'ultima è più dotata in termini di spazio di archiviazione) in maniera efficiente e con salvaguardia della privacy rappresenta un argomento insteressante da indagare.

\paragraph{Governance auto-adattativa}
Mentre la blockchain IoTeX offre incentivi per mantenere il consenso sui suoi registri, per il momento non dispone di un meccanismo in grado di auto-modificare le regole che governano il suo protocollo. Per affrontare ciò, abbiamo in programma di condurre ricerche su governance e auto-modifica.

\paragraph{Blockchain con struttura ad albero}
L'attuale blockchain IoTeX ha due livelli e, naturalmente, potrebbe essere estesa a un albero di blockchain sfruttando tecniche simili a quelle utilizzate in Plasma e Cosmo. Il piano è quello di valutare queste proposte e migliorare l'attuale progetto di IoTeX, ed alla fine supportare strutture gerarchiche più complesse.

\section{Conclusioni}
In questo Whitepaper, abbiamo introdotto IoTeX, una blockchain scalabile, privata, ed estensibile, dedicata all'Internet of Things, con la sua architettura e le sue tecnologie di base, che includono:

1. Blockchain in blockchain, per massimizzare scalabilità e privacy, 2. privacy totale su blockchain basata su codice di pagamento inoltrabile, ring signature a dimensione costante e senza configurazione trusted, e implementazione iniziale di bulletproofs, 3. consenso rapido con finalità istantanea basata su VRF e PoS per volumi di transazioni elevati e finalità istantanea e 4. architetture di sistema basate su IoTeX flessibili e leggere.

\section{Ringraziamenti}
Vogliamo esprimere la nostra gratitudine ai nostri mentori e consulenti e alle molte persone nelle comunità dell'IoT, della crittografia e delle criptovalute per i loro feedback iniziali e i suggerimenti costruttivi.

\pagebreak

\begin{otherlanguage}{english}
	\bibliographystyle{plain}
	\bibliography{bibliography}
\end{otherlanguage}



\end{document}