\documentclass[a4paper,12pt]{article}
\usepackage{amsmath}
\usepackage{graphicx}
\usepackage{hyperref}
\usepackage[latin1]{inputenc}

\title{IoTeX
\linebreak
\Large Una Rete Decentralizzata per l'Internet of Things
\linebreak
\Large Basata su una Blockchain Incentrata sulla Privacy}
\author{Il Team IoTeX (support@iotex.io)}
\date{Ultimo Aggiornamento: 23 Marzo, 2018
\linebreak Version 1.4}


\begin{document}

\maketitle

\vspace{120}


\textbf{Disclaimer} Questo documento deve essere inteso come una panoramica tecnica. Non vuole essere omnicomprensivo, né rappresentare un progetto definitivo; pertanto aspetti secondari non vengono trattati, come API, interconnessioni o linguaggi di programmazione.  

\pagebreak

\begin{abstract}
The majority of Internet of Things (IoT) devices are developed in a centralized
La maggior parte dei dispositivi IoT (Internet of Things), sebbene decentralizzati per natura, ad oggi sono distribuiti in modo centralizzato. Molti problemi sono emersi: scalabilità, costi operativi elevati, problemi di privacy, rischi per la sicurezza, e mancanza di valori funzionali. La Blockchain, decentralizzata per definizione, potrebbe essere una buona soluzione a questi problemi. Innanzitutto, la blockchain è elastica, abbastanza da risolvere la sfida della scalabilità dell'IoT in modo economicamente vantaggioso. In secondo luogo, mantenendo i dati all'interno di blockchain ben definite si elimina la preoccupazione per i dati IoT memorizzati in cloud potenzialmente suscettibili di trapelare o di abusi. In terzo luogo, le blockchain con smart contract e token hanno un enorme potenziale per rendere possibile il coordinamento autonomo di dispositivi per creare valore funzionale. Tuttavia, le blockchain esistenti hanno i loro limiti nell'affrontare i problemi dell'IoT, a causa delle sue caratteristiche peculiari, ad esempio la grande quantità ed eterogeneità dei dispositivi, i limiti nella potenza di elaborazione, nell'archiviazione dati, nell'alimentazione, ecc. 
Questo documento introduce IoTeX, una rete decentralizzata per l'IoT basata su una blockchain incentrata sulla privacy con quattro importanti innovazioni:

\begin{itemize}

\item{Blockchains in blockchain per una rete distribuita ben bilanciata che massimizza la scalabilità e la privacy in modo economicamente vantaggioso;}

\item{La vera privacy sulla blockchain basata sul codice di pagamento inoltro, costante- firma dell'anello di dimensioni senza configurazione attendibile e prima implementazione di antiproiettile;}

\item{Consenso rapido con finalità immediate che migliorano notevolmente il rendimento di la rete e riducendo i costi di transazione;}

\item{Architetture di sistema basate su IoTeX flessibili e leggere per le applicazioni IoT chiave in più settori industriali.}
\end{itemize}

\end{abstract}

\end{document}
