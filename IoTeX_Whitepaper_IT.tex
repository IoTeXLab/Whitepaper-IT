\documentclass[a4paper,12pt,draft]{article}
\usepackage[a4paper]{geometry}
\usepackage[italian]{babel}
\usepackage{amsmath}
\usepackage{graphicx}
\usepackage{hyperref}
\usepackage[utf8]{inputenc}
\usepackage{bookmark}

\geometry{
 left=33mm,
 right=30mm
 }

\title{IoTeX
\linebreak
\Large Una Rete Decentralizzata per l'Internet of Things
\linebreak
\Large Basata su una Blockchain Incentrata sulla Privacy}
\author{Il Team IoTeX (support@iotex.io)}
\date{Ultimo Aggiornamento: 23 Marzo, 2018
\linebreak Version 1.4}


\begin{document}

\maketitle

\vspace{120pt}


\textbf{Disclaimer} Questo documento deve essere inteso come una panoramica tecnica. Non vuole essere esaustivo, né rappresentare un progetto definitivo; pertanto aspetti secondari, come API, interconnessioni o linguaggi di programmazione, non vengono trattati.

\pagebreak

\begin{abstract}
	La maggior parte dei dispositivi IoT (Internet of Things), sebbene decentralizzati per natura, ad oggi sono distribuiti in modo centralizzato. Molti problemi sono emersi: scalabilità, costi operativi elevati, problemi di privacy, rischi per la sicurezza, e mancanza di valore funzionale. La Blockchain, decentralizzata per definizione, può rappresentare una buona soluzione a questi problemi. Innanzitutto, la blockchain è abbastanza elastica da risolvere la sfida della scalabilità dell'IoT in modo economicamente vantaggioso. In secondo luogo, mantenendo i dati all'interno di blockchain ben definite, si eliminano i timori per i dati IoT memorizzati in cloud, potenzialmente suscettibili di trapelare o di essere violati. In terzo luogo, le blockchain con smart contract e token hanno un enorme potenziale per consentire il coordinamento autonomo dei dispositivi al fine di creare valore funzionale. Tuttavia, le blockchain esistenti hanno i loro limiti nell'affrontare i problemi dell'IoT, a causa delle caratteristiche peculiari che lo contraddistinguono, ad esempio la grande quantità e l'eterogeneità dei dispositivi, i limiti nella potenza di elaborazione, nell'archiviazione dati, nell'alimentazione, ecc.
	Questo documento introduce IoTeX, una rete decentralizzata per l'IoT basata su una blockchain incentrata sulla privacy, con quattro importanti innovazioni:

	\begin{itemize}

		\item
		      Blockchains in blockchain per una rete distribuita ben bilanciata che massimizza la scalabilità e la privacy in modo economicamente vantaggioso;

		\item
		      La vera privacy sulla blockchain basata sul codice di pagamento inoltro, costante- firma dell'anello di dimensioni senza configurazione attendibile e prima implementazione di antiproiettile;

		\item
		      Consenso rapido con finalità immediate che migliorano notevolmente il rendimento di la rete e riducendo i costi di transazione;

		\item
		      Architetture di sistema basate su IoTeX flessibili e leggere per le applicazioni IoT chiave in più settori industriali.

	\end{itemize}

\end{abstract}

\pagebreak

\tableofcontents

\pagebreak

\section{L'Internet of Things}
L'Internet of Things (IoT) sta emergendo rapidamente come manifestazione della visione di una società collegata in rete: qualsiasi cosa che beneficia di una connessione, è connesso. Eppure, questa trasformazione su vasta scala rappresenta solo l'inizio. Il numero di dispositivi IoT è destinato a crescere del 21\% ogni anno, raggiungendo i 18 miliardi nel 2022 \cite{10}, mentre il mercato globale dell'IoT è destinato a passare dai 170 miliardi di dollari del 2017 a 560 miliardi di dollari entro il 2022 \cite{15}, con un tasso di crescita annua del 26,9\%. Sebbene molti esperti dell'industria e consumatori entusiasti hanno definito l'IoT come la prossima rivoluzione industriale o il prossimo internet, ci sono tre problemi principali che frenano in maniera massiccia lo sviluppo e l'adozione dell'IoT.

\subsection{Il problema della scalabilità}
La maggior parte dei dispositivi IoT sono ad oggi connessi e controllati in maniera centralizzata. I dispositivi IoT sono connessi ad infrastrutture di back-end, su servizi cloud pubblici oppure localmente all'interno di server farm, per trasmettere dati oppure ricevere comandi di controllo.
Attualmente, la dimensione dell IoT è strozzata dal livello di scalabilità ed elasticità di queste infrastrutture di back-end, server e data center. E' improbabile che il costo operativo sostanzialmente elevato necessario per scalare l'IoT sia coperto dai profitti della vendita dei dispositivi. Di conseguenza, molti fornitori IoT non riescono a proporre dispositivi economicamente vantaggiosi ed applicazioni che siano abbastanza scalabili ed affidabili per scenari reali.

\subsection{Mancanza di Privacy}
Si prevede che l'IoT permetterà la partecipazione di massa degli utenti finali a servizi mission critical come l'energia, la  mobilità, la stabilità legale e democratica. Le sfide per la privacy hanno origine dal fatto che l'IoT interagisce con il mondo fisico in modi diretti e automatici, e la quantità di dati raccolti aumenterà notevolmente quando si ridimensionerà. Alcune  delle minacce alla privacy comuni, come elencate in \cite{37}, sono:

\begin{enumerate}

	\item
	      Identificazione: associare un identificatore  persistente), ad es. Un nome e un indirizzo o uno pseudonimo di qualsiasi tipo, con un individuo;
	\item
	      Localizzazione e tracciamento: ottenere la posizione di un individuo attraverso diversi mezzi;
	\item
	      Profilazione: Compilare fascicoli informativi sulle persone per dedurrne gli interessi per
	      associazione con altri profili e fonti di dati;
	\item
	      Interazione e presentazione che violano la privacy: trasmettere informazioni private attraverso un mezzo pubblico e nel processo rivelarle ad un pubblico indesiderato;
	\item
	      Transizioni del ciclo di vita: i dispositivi spesso memorizzano enormi quantità di dati sulla propria storia durante l'intero ciclo di vita che potrebbe trapelare durante i cambiamenti della sfera di controllo nel ciclo di vita di un dispositivo;
	\item
	      Attacco all'inventario: raccolta non autorizzata di informazioni sull'esistenza e sulle caratteristiche degli oggetti personali, ad esempio, i ladri d'appartamento potrebbero utilizzare l'inventario dati per controllare la proprietà, e individuare un momento sicuro per entrare;
	\item
	      Collegamento: collegamento di sistemi diversi precedentemente separati in modo tale che la combinazione delle fonti di dati riveli informazioni (vere o errate) che il soggetto non aveva rivelato alle fonti isolate e, soprattutto, che non intendeva rivelare.
\end{enumerate}
Tutte queste tipiche minacce alla privacy sono dovute a perdite di dati a livello del dispositivo, oppure alla perdita di dati durante la comunicazione, o più spesso alla perdita di dati nella parte centralizzata della rete.

\subsection{Mancanza di Valore Funzionale}
La maggior parte delle soluzioni IoT esistenti non crea valore significativo. Il solo "Essere connesso" rappresenta la proposta di valore più utilizzata. Tuttavia, abilitare semplicemente la connettività non rende un dispositivo intelligente o utile. La maggior parte del valore che l'IoT produce è dovuto all'interazione, alla cooperazione, ed infine dal coordinamento autonomo di entità eterogenee. Alcune buone analogie sono le singole cellule che cooperano per costruire gli organismi multicellulari, gli insetti che insieme costruiscono società, oppure gli uomini che costituiscono città e stati. Grazie alla cooperazione, tutti questi individui si uniscono per costruire qualcosa che ha un valore maggiore rispetto a tutti loro isolatamente. Sfortunatamente, secondo \cite{29}, l'85\% dei dispositivi obsoleti non hanno la capacità di interagire o cooperare tra loro, a causa di problemi di compatibilità. La condivisione dei dati per il business e le indizazioni operative è quasi impossibile.

\section{Blockchain}
La tecnologia della Blockchain è stata introdotta nel 2008 e la sua prima implementazione, ovvero Bitcoin, è stata introdotta un anno dopo, nel 2009, pubblicata nel documento \emph{Bitcoin: A Peer-to-Peer Electronic Cash System} \cite{21} di Satoshi Nakamoto (pseudonimo). Essenzialmente, la blockchain è un database transazionale distribuito, condiviso tra tutti i nodi partecipanti nella rete. Questa è la principale innovazione tecnica di Bitcoin, ed agisce come un registro pubblico per le transazioni. Ogni nodo nel sistema ha una copia completa dello stato attuale della blockchain, che contiene ogni transazione che sia mai stata eseguita. Ogni blocco della blockchain contiene un hash del blocco precedente, che collega i due blocchi insieme. Tutti i collegati tra loro diventano una blockchain.

\subsection{Ingredienti}
Una blockchain può essere percepita come un continuum tetra-dimensionale avente tre livelli orizzontali che includono transazioni e blocchi, consenso, interfaccia di calcolo, e amministrazione (\emph{governance}), l'unico livello verticale.

\subsubsection{Transazioni e blocchi}
Essendo il livello orizzontale più in basso, le transazioni firmate vengono trasmesse tra tutti i nodi e i blocchi vengono generati dai nodi completi (\emph{full nodes}). Questa e la base della blockchain dove il trasferimento di beni digitali (e dunque il valore a loro associato) e la sicurezza degli account si ottengono attraverso primitive come firma a curva ellittica, funzione hash e Merkle tree.

\subsubsection{Consenso}
Il livello orizzontale intermedio mostra la natura peer-to-peer della blockchain, dove tutti i nodi all'interno della rete raggiungono il consenso su tutti gli stati interni della catena  attraverso tecniche come Proof of Work (PoW), Proof of Stake (PoS) e le loro varianti, Byzantine fault tolerance (BFT) a le sue varianti, \emph{etc}. Il livello di consenso interessa principalmente la scalabilità. Il PoW viene tipicamente considerato meno scalabile rispetto al PoS. Inoltre, questo livello ha un forte impatto sulla sicurezza in termini di doppia spesa ed altri attacchi che si concentrano sul modificare gli stati della blockchain in modi inattesi.

\subsubsection{Interfaccia di calcolo}
I primi due livelli orizzontali realizzano la forma di una blockchain mentre il livello di interfaccia di calcolo è fondamentale per rendere utile una blockchain, il che comprende estendibilità ed usabilità. Ad esempio, Ethereum ha implementato un lo smart contract per
consentire la programmabilità in modo da poter contare sul "computer mondiale" distribuito per l'esecuzione dei termini di un contratto. Anche il sidechain, insieme con il mining congiunto, sono stati sviluppati in modo intensivo per supportare la programmabilità. In protocolli di secondo livello come la rete Raiden \cite{25}, è stato sviluppato il canale di stato per estendere la scalabilità della blockchain a questo livello. Inoltre, anche strumenti, SDK, framework ed interfacce grafiche sono
estremamente importanti per l'usabilità. Il livello di interfaccia di calcolo offre agli sviluppatori la possibilità di sviluppare app decentralizzate (DApps), una parte essenziale per rendere la blockchain utile e preziosa.

\subsubsection{Amministrazione}
Come per gli organismi, le blockchain di maggior successo saranno quelle che riusciranno ad adattarsi meglio al loro ambiente. Supponendo che questi sistemi debbano evolversi per sopravvivere, il progetto iniziale è importante ma, nel lungo termine, i meccanismi per il cambiamento
sono più importanti, ed essi sono noti come il livello verticale di amministrazione (\emph{governance}). Ci sono due componenti critici della governance:
\begin{itemize}
	\item
	      Incentivo: ogni gruppo nel sistema ha i propri incentivi. Gli incentivi non sono sempre allineati al 100\% con tutti gli altri gruppi nel sistema. I gruppi proporranno nel tempo cambiamenti che sono vantaggiosi per loro. Gli organismi sono di parte rispetto alla propria sopravvivenza. Ciò normalmente si manifesta in cambiamenti nella struttura retributiva, nella politica monetaria o in equilibri di potere.

	\item
	      Coordinamento: poiché è improbabile che tutti i gruppi abbiano un allineamento sugli incentivi del 100\% tutte le volte, la capacità di ciascun gruppo di coordinarsi attorno ai propri incentivi comuni è fondamentale per loro per produrre un cambiamento. Se un gruppo riesce a  coordinarsi meglio di un altro, crea uno squilibrio di potere a proprio favore. In pratica, un fattore decisivo è quanto coordinamento si può realizzare sulla blockchain (ad es. voti alle regole del sistema come Tezos \cite{34}, o addirittura ripristinare il registro se gli azionisti di maggioranza non gradiscono una modifica) rispetto a quello realizzabile fuori dalla blockchain (come per i Bitcoin Improvement Proposals (BIPs)
	      \cite{3}).

\end{itemize}



\pagebreak

\bibliography{bibliography}

\bibliographystyle{plain}

\end{document}