 %!BIB program = bibtex
\documentclass[a4paper,12pt]{article}
\usepackage{amsmath}
\usepackage{graphicx}
\usepackage{hyperref}
\usepackage[utf8]{inputenc}
\usepackage{bookmark}


\title{IoTeX
\linebreak
\Large Una Rete Decentralizzata per l'Internet of Things
\linebreak
\Large Basata su una Blockchain Incentrata sulla Privacy}
\author{Il Team IoTeX (support@iotex.io)}
\date{Ultimo Aggiornamento: 23 Marzo, 2018
\linebreak Version 1.4}


\begin{document}

\maketitle

\vspace{120pt}


\textbf{Disclaimer} Questo documento deve essere inteso come una panoramica tecnica. Non vuole essere omnicomprensivo, né rappresentare un progetto definitivo; pertanto aspetti secondari non vengono trattati, come API, interconnessioni o linguaggi di programmazione.

\pagebreak

\begin{abstract}
    \begin{center}
        \textbf{Abstract}
    \end{center}
	La maggior parte dei dispositivi IoT (Internet of Things), sebbene decentralizzati per natura, ad oggi sono distribuiti in modo centralizzato. Molti problemi sono emersi: scalabilità, costi operativi elevati, problemi di privacy, rischi per la sicurezza, e mancanza di valore funzionale. La Blockchain, decentralizzata per definizione, può rappresentare una buona soluzione a questi problemi. Innanzitutto, la blockchain è abbastanza elastica da risolvere la sfida della scalabilità dell'IoT in modo economicamente vantaggioso. In secondo luogo, mantenendo i dati all'interno di blockchain ben definite, si eliminano i timori per i dati IoT memorizzati in cloud, potenzialmente suscettibili di trapelare o di essere violati. In terzo luogo, le blockchain con smart contract e token hanno un enorme potenziale per consentire il coordinamento autonomo dei dispositivi al fine di creare valore funzionale. Tuttavia, le blockchain esistenti hanno i loro limiti nell'affrontare i problemi dell'IoT, a causa delle caratteristiche peculiari che lo contraddistinguono, ad esempio la grande quantità e l'eterogeneità dei dispositivi, i limiti nella potenza di elaborazione, nell'archiviazione dati, nell'alimentazione, ecc.
	Questo documento introduce IoTeX, una rete decentralizzata per l'IoT basata su una blockchain incentrata sulla privacy, con quattro importanti innovazioni:

	\begin{itemize}

		\item{Blockchains in blockchain per una rete distribuita ben bilanciata che massimizza la scalabilità e la privacy in modo economicamente vantaggioso;}

		\item{La vera privacy sulla blockchain basata sul codice di pagamento inoltro, costante- firma dell'anello di dimensioni senza configurazione attendibile e prima implementazione di antiproiettile;}

		\item{Consenso rapido con finalità immediate che migliorano notevolmente il rendimento di la rete e riducendo i costi di transazione;}

		\item{Architetture di sistema basate su IoTeX flessibili e leggere per le applicazioni IoT chiave in più settori industriali.}

	\end{itemize}

\end{abstract}

\pagebreak

\tableofcontents


\section{L'Internet of Things}
L'Internet of Things (IoT) sta emergendo rapidamente come manifestazione della visione di una società collegata in rete: qualsiasi cosa che beneficia di una connessione, è connesso. Eppure, questa trasformazione su vasta scala rappresenta solo l'inizio. Il numero di dispositivi IoT è destinato a crescere del 21\% ogni anno, raggiungendo i 18 miliardi nel 2022 \cite{r10}, mentre il mercato globale dell'IoT è destinato a passare dai 170 miliardi di dollari del 2017 a 560 miliardi di dollari entro il 2022 \cite{r15}, con un tasso di crescita annua del 26,9\%. Sebbene molti esperti dell'industria e consumatori entusiasti hanno definito l'IoT come la prossima rivoluzione industriale o il prossimo internet, ci sono tre problemi principali che frenano in maniera massiccia lo sviluppo e l'adozione dell'IoT.

\section{Il problema della scalabilità}
La maggior parte dei dispositivi IoT sono ad oggi connessi e controllati in maniera centralizzata. I dispositivi IoT sono connessi ad infrastrutture di back-end, su servizi cloud pubblici oppure localmente all'interno di server farm, per trasmettere dati oppure ricevere comandi di controllo.
Attualmente, la dimensione dell IoT è strozzata dal livello di scalabilità ed elasticità di queste infrastrutture di back-end, server e data center. E' improbabile che il costo operativo sostanzialmente elevato necessario per scalare l'IoT sia coperto dai profitti della vendita dei dispositivi. Di conseguenza, molti fornitori IoT non riescono a proporre dispositivi economicamente vantaggiosi ed applicazioni che siano abbastanza scalabili ed affidabili per scenari reali.

\pagebreak

\bibliography{bibliography}

\bibliographystyle{plain}

\end{document}